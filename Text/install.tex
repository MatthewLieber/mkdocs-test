\section{Installation Instructions}
\label{sec:install}

The MVAPICH2 installation process is designed to enable the most widely utilized
features on the target build OS by default. 
%Supported operating systems include Linux and Solaris. The default interface is
%OFA-IB-CH3/OFA-IWARP-CH3 on Linux and uDAPL on Solaris.
The other interfaces, as indicated in Figure~\ref{fig:modules}, can also be
selected on Linux.  This installation section provides generic instructions for
building from a tarball or our latest sources. 

{\em In order to obtain best performance and scalability while having
flexibility to use a large number of features, the MVAPICH team strongly
recommends the use of following interfaces for different adapters: 1) OFA-IB-CH3
interface for all Mellanox InfiniBand adapters, 2) TrueScale (PSM-CH3) interface for all
Intel InfiniBand adapters, 3) OFA-RoCE-CH3 interface for all RoCE adapters, 4)
OFA-iWARP-CH3 for all iWARP adapters and 5) Shared-Memory-CH3 for single node
SMP system and laptop.}

Please see the appropriate subsection for specific configuration 
instructions for the interface-adapter you are targeting. 

\subsection{Building from a tarball}
The MVAPICH2 \mvapichrcversion source code package includes MPICH
\mpichversion.  All the required files are present as a single tarball.
Download the most recent distribution tarball from:\\
~\href{http://mvapich.cse.ohio-state.edu/downloads}{http://mvapich.cse.ohio-state.edu/downloads}

Unpack the tarball and use the standard GNU procedure to compile:

\CommandBox{
\$ tar -xzf mvapich2-\mvapichrcversion.tgz \\
\$ cd mvapich2-\mvapichrcversion \\
\$ ./configure \\
\$ make \\
\$ make install}{0.7}

We now support parallel make and you can use the -j$<$num threads$>$ option to
speed up the build process.  You can use the following example to 
spawn 4 threads instead of the preceding make step.

\CommandBox{\$ \mbox{make -j 4}}{0.7} \\

In order to install a debug build, please use the following
configuration option. \textit{Please note that using debug builds may impact
performance.}

\CommandBox{
\$ ./configure --enable-g=all --enable-error-messages=all \\
\$ make \\
\$ make install}{0.7}

\subsection{Obtaining and Building the Source from SVN repository}
These instructions assume you have already installed subversion.

The MVAPICH2 SVN repository is available at: \\
~\href{https://scm.nowlab.cse.ohio-state.edu/svn/mpi/mvapich2/}{https://scm.nowlab.cse.ohio-state.edu/svn/mpi/mvapich2/}

Please keep in mind the following guidelines before deciding which 
version to check out:

\begin{itemize}

%\item ``tags/\mvapichversion'' is the exact version released with no updates 
\item ``tags/\mvapichrcversion'' is the exact version released with no updates 
for bug fixes or new features.

\begin{itemize}
\item To obtain the source code from tags/\mvapichrcversion:

\CommandBox{\$ svn co \\
https://scm.nowlab.cse.ohio-state.edu/svn/mpi/mvapich2/tags/\mvapichrcversion \\
mvapich2}{0.9}

\end{itemize}

\begin{comment}
\item ``branches/\mvapichbranchversion'' is a stable version with bug fixes.  New features are not added to this branch.

\begin{itemize}
\item To obtain the source code from branches/\mvapichbranchversion:

\CommandBox{\$ svn co
https://scm.nowlab.cse.ohio-state.edu/svn/mpi/mvapich2/branches/\mvapichbranchversion \\
mvapich2}{0.9}

\end{itemize}
\end{comment}

\item ``trunk'' will contain the latest source code as we enhance 
and improve MVAPICH2.  It may contain newer
features and bug fixes, but is lightly tested.

\begin{itemize}
\item To obtain the source code from trunk:

\CommandBox{\$ svn co
    https://scm.nowlab.cse.ohio-state.edu/svn/mpi/mvapich2/trunk \\
mvapich2}{0.9}

\end{itemize}

\begin{comment}
        \item \emph{To follow the exact steps indicated in this user
        guide, dated June 13th, 2008 or later, please download the
        latest source from the 1.5 branch}
\end{comment}
\end{itemize}

The mvapich2 directory under your present working directory contains a working copy of the \mbox{MVAPICH2} source code.
Now that you have obtained a copy of the source code, you need to update the files in the source tree:

\CommandBox{\$ cd mvapich2 \\
\$ ./autogen.sh}{0.7}

This script will generate all of the source and configuration files you
need to build MVAPICH2. You will need \texttt{autoconf} version $>=$ 2.67,
\texttt{automake} version $>=$ 1.12.3, \texttt{libtool} version $>=$ 2.4

%If the command
%``autoconf'' on your machine does not run autoconf 2.63 or later, but you do have a new enough autoconf available, 
%then you can specify the correct one with the AUTOCONF environment variable (the AUTOHEADER environment variable
%is similar).  Once you've prepared the working copy by running maint/updatefiles, just follow the usual configuration
%and build procedure:

\CommandBox{\$ ./configure \\
\$ make \\
\$ make install }{0.7}

\subsection{Selecting a Process Manager}

MVAPICH2 provides the mpirun\_rsh/mpispawn framework from MVAPICH
distribution. Using mpirun\_rsh should provide the fastest startup of
your MPI jobs. More details can be found in
Section~\ref{sec:run-mpirun-rsh}. In addition, MVAPICH2 also includes
the Hydra process manager from MPICH-\mpichversion. For more details on using
Hydra, please refer to Section~\ref{sec:run-hydra}.

By default, \emph{mpiexec} uses the Hydra process launcher. Please note
that neither mpirun\_rsh, nor Hydra require you to start daemons in
advance on the nodes used for a MPI job.
Both mpirun\_rsh and Hydra can be used with any of the eight 
interfaces of this MVAPICH2 release, as indicated in 
Figure~\ref{fig:modules}.

\newpage
\subsubsection{Customizing Commands Used by mpirun\_rsh}
Usage: ./configure [OPTION]... [VAR=VALUE]...

To assign environment variables (e.g., CC, CFLAGS...), specify them as
VAR=VALUE.  See below for descriptions of some of the useful variables.

\begin{tabular}{l l}
RSH\_CMD     & path to rsh command \\
SSH\_CMD     & path to ssh command \\
ENV\_CMD     & path to env command \\
DBG\_CMD     & path to debugger command \\
XTERM\_CMD   & path to xterm command \\
SHELL\_CMD   & path to shell command \\
TOTALVIEW\_CMD & path to totalview command \\
\end{tabular}

\subsubsection{Using SLURM}
\label{subsec:config-slurm}

If you'd like to use SLURM to launch your MPI programs please use the following
configure options.
% There is now a configuration option that can be used to allow mpicc and the
% other MPI compiler commands to automatically link MPI programs to the SLURM's
% PMI library.

To configure MVAPICH2 to use PMI-1 support in SLURM:

\CommandBox{\$ ./configure --with-pmi=pmi1 --with-pm=slurm}{0.7}

To configure MVAPICH2 to use PMI-2 support in SLURM:

\CommandBox{\$ ./configure --with-pmi=pmi2 --with-pm=slurm}{0.7}

To configure MVAPICH2 to use PMIx support in SLURM:

\CommandBox{\$ ./configure --with-pmi=pmix --with-pm=slurm}{0.7}

\subsubsection{Using SLURM with support for PMI Extensions}
\label{subsec:config-slurm-pmix}

MVAPICH2 automatically detects and uses PMI extensions if available from the
process manager. To build and install SLURM with PMI support, please follow
these steps:

Download the SLURM source tarball for SLURM-15.08.8 from \\
~\href{http://slurm.schedmd.com/download.html}{http://slurm.schedmd.com/download.html}.

Download the patch to add PMI Extensions in SLURM from \\
~\href{http://mvapich.cse.ohio-state.edu/download/mvapich/osu-shmempmi-slurm-15.08.8.patch}{http://mvapich.cse.ohio-state.edu/download/\linebreak[0]mvapich/osu-shmempmi-slurm-15.08.8.patch}.

\CommandBox{\$ tar -xzf slurm-15.08.8.tar.gz \\
\$ cd slurm-15.08.8 \\
\$ patch -p1 < osu-shmempmi-slurm-15.08.8.patch \\
\$ ./configure --prefix=/path/to/slurm/install --disable-pam\\
\$ make -j4 \&\& make install \&\& make install-contrib}{0.7}

To configure MVAPICH2 with the modified SLURM, please use:

\CommandBox{\$ ./configure --with-pm=slurm --with-pmi=pmi2 --with-slurm=/path/to/slurm/install}{0.7}

MVAPICH2 can also be configured with PMIx plugin of SLURM:

\CommandBox{\$ ./configure --with-pm=slurm --with-pmi=pmix --with-pmix=/path/to/pmix/install}{0.7}

Note that --with-pmix should refer to the pmix/install directory that is used to build SLURM.

Please refer to Section~\ref{sec:run-slurm} for information on how to run MVAPICH2 using
SLURM.


\subsubsection{Using Job Step Manager (JSM)}
\label{subsec:config-jsm}

MVAPICH2 supports PMIx extensions for JSM. To configure MVAPICH2 with PMIx plugin of JSM, please use:

\CommandBox{\$ ./configure --with-pm=jsm --with-pmi=pmix --with-pmix=/path/to/pmix/install}{0.7}

Note that --with-pmix should refer to the pmix/install directory that is used to build JSM.

MVAPICH2 can also use pmi4pmix library to support JSM. It can be configured as follows:

\CommandBox{\$ ./configure --with-pm=jsm --with-pmi=pmi4pmix \
    --with-pmi4pmix=/path/to/pmi4pmix/install}{0.7}

Please refer to Section~\ref{sec:run-jsm} for information on how to run MVAPICH2
using the Jsrun launcher.

\subsubsection{Using Flux Resource Manager}
\label{subsec:config-flux}

To configure MVAPICH2 with Flux support, please use:

\CommandBox{\$ ./configure --with-pm=flux --with-flux=/path/to/flux/install}{0.7}

Please refer to Section~\ref{sec:run-flux} for information on how to run MVAPICH2
using Flux.

\subsection{Configuring a build for OFA-IB-CH3/OFA-iWARP-CH3/OFA-RoCE-CH3}
\label{subsec:config-gen2}


OpenFabrics (OFA) IB/iWARP/RoCE with the CH3 channel
is the default interface on Linux.  
It can be explicitly selected by configuring with:

\CommandBox{\$ ./configure --with-device=ch3:mrail --with-rdma=gen2}{0.7}

Both static and shared libraries are built by default. In order to build with static libraries only, configure as follows:

\CommandBox{\$ ./configure --with-device=ch3:mrail --with-rdma=gen2 --disable-shared}{0.8}

To enable use of the TotalView debugger, the library needs to be configured
in the following manner:

\CommandBox{\$ ./configure --with-device=ch3:mrail --with-rdma=gen2 --enable-g=dbg \
                           --enable-debuginfo}{0.9}

Configuration Options for OpenFabrics IB/iWARP/RoCE

\begin{itemize}
	\item Configuring with Shared Libraries
		\begin{itemize}
			\item Default: Enabled
			\item Enable:  \texttt{--enable-shared}
            \item Disable: \texttt{--disable-shared}
		\end{itemize}

	\item Configuring with TotalView support
		\begin{itemize}
			\item Default: Disabled
			\item Enable: \texttt{--enable-g=dbg \\
				--enable-debuginfo}
		\end{itemize}

	\item Path to OpenFabrics Header Files
		\begin{itemize}
			\item Default: Your PATH
			\item Specify: \texttt{--with-ib-include=path}
		\end{itemize}

	\item Path to OpenFabrics Libraries
		\begin{itemize}
			\item Default: The systems search path for libraries.
			\item Specify: \texttt{--with-ib-libpath=path}
		\end{itemize}

	\item Support for Hybrid UD-RC/XRC transports
        \begin{itemize}
            \item Default: Disabled
            \item Enable:  \texttt{--enable-hybrid}
		\end{itemize}

	\item Support for RDMA CM
		\begin{itemize}
			\item Default: enabled, except when BLCR support is enabled
			\item Disable: \texttt{--disable-rdma-cm}
		\end{itemize}

	\item Support for RoCE
		\begin{itemize}
			\item Default: enabled
		\end{itemize}

	\item Registration Cache
		\begin{itemize}
			\item Default: enabled
			\item Disable: \texttt{--disable-registration-cache}
		\end{itemize}

	\item ADIO driver for Lustre:
		\begin{itemize}
			\item  When compiled with this support, MVAPICH2 will use the optimized
				driver for Lustre. In order to enable this feature, the flag \\
				\texttt{--enable-romio --with-file-system=lustre} \\
				 should be passed to \texttt{configure}
				(\texttt{--enable-romio} is optional as it is enabled by default).
		         You can add support for more file systems using \\
				\texttt{--enable-romio --with-file-system=lustre+nfs+pvfs2}
		\end{itemize}

	\item LiMIC2 Support
		\begin{itemize}
			\item Default: disabled
			\item Enable:\\ \texttt{--with-limic2[=$<$path to LiMIC2 installation$>$]\\
			--with-limic2-include=$<$path to LiMIC2 headers$>$\\
			--with-limic2-libpath=$<$path to LiMIC2 library$>$}
		\end{itemize}

        \item CMA Support
            \begin{itemize}
                \item Default: enabled
                \item Disable: \texttt{--without-cma}
            \end{itemize}

	\item Header Caching
		\begin{itemize}
			\item Default: enabled
			\item Disable: \texttt{--disable-header-caching}
		\end{itemize}

    \item MPI Tools Information Interface (MPI-T) Support
        \begin{itemize}
            \item Default: disabled
            \item Enable: \texttt{--enable-mpit-pvars}
        \end{itemize}
    
    \item Checkpoint/Restart
        \begin{itemize}
          \item Option name: \texttt{--enable-ckpt}
          \item Require: Berkeley Lab Checkpoint/Restart (BLCR)
          \item Default: disabled
        \end{itemize}

        The Berkeley Lab Checkpoint/Restart (BLCR) installation is automatically
        detected if installed in the standard location. To specify an alternative
        path to the BLCR installation, you can either use:\\

        \texttt{--with-blcr=<path/to/blcr/installation> } \\
        or \\
        \texttt{--with-blcr-include=<path/to/blcr/headers> \\
                --with-blcr-libpath=<path/to/blcr/library>}


    \item Checkpoint Aggregation
        \begin{itemize}
          \item Option name: \texttt{--enable-ckpt-aggregation} or \texttt{--disable-ckpt-aggregation}
          \item Automatically enable Checkpoint/Restart
          \item Require: Filesystem in Userspace (FUSE)
          \item Default: enabled (if Checkpoint/Restart enabled and FUSE is present)
        \end{itemize}

        The Filesystem in Userspace (FUSE) installation is automatically detected
        if installed in the standard location. To specify an alternative path to
        the FUSE installation, you can either use:\\

        \texttt{--with-fuse=<path/to/fuse/installation> } \\
        or \\
        \texttt{--with-fuse-include=<path/to/fuse/headers> \\
                --with-fuse-libpath=<path/to/fuse/library>}

        \item Application-Level and Transparent System-Level Checkpointing with SCR
        \begin{itemize}
            \item Option name: \texttt{--with-scr}
            \item Default: disabled
        \end{itemize}

        SCR caches checkpoint data in storage on the compute nodes of a Linux
        cluster to provide a fast, scalable checkpoint / restart capability for MPI
        codes.

    \item Process Migration
        \begin{itemize}
          \item Option name: \texttt{--enable-ckpt-migration}
          \item Automatically enable Checkpoint/Restart
          \item Require: Fault Tolerance Backplane (FTB)
          \item Default: disabled
        \end{itemize}

        The Fault Tolerance Backplane (FTB) installation is automatically detected
        if installed in the standard location. To specify an alternative path to
        the FTB installation, you can either use:\\

        \texttt{--with-ftb=<path/to/ftb/installation> } \\
        or \\
        \texttt{--with-ftb-include=<path/to/ftb/headers> \\
                --with-ftb-libpath=<path/to/ftb/library>}


	\item eXtended Reliable Connection
		\begin{itemize}
			\item Default: enabled (if OFED installation supports it)
			\item Enable: \texttt{--enable-xrc}
			\item Disable: \texttt{--disable-xrc}
		\end{itemize}

	\item HWLOC Support (Affinity)
		\begin{itemize}
			\item Default: enabled
			\item Disable: \texttt{--without-hwloc}
		\end{itemize}
	\item Support for 64K or greater number of cores
		\begin{itemize}
			\item Default: 64K or lower number of cores
			\item Enable: \texttt{--with-ch3-rank-bits=32}
		\end{itemize}

\end{itemize}


\subsection{Configuring a build for NVIDIA GPU with OFA-IB-CH3}
\label{subsec:config-cuda-gen2}

This section details the configuration option to enable GPU-GPU
communication with the OFA-IB-CH3 interface of the MVAPICH2 MPI library.
For more options on configuring the OFA-IB-CH3 interface, please refer to
Section~\ref{subsec:config-gen2}.

\begin{itemize}
    \item Default: disabled
    \item Enable: \texttt{--enable-cuda}
\end{itemize}

The CUDA installation is automatically detected if installed in the standard
location. To specify an alternative path to the CUDA installation, you can
either use:\\
\texttt{--with-cuda=<path/to/cuda/installation> } \\
or \\
\texttt{--with-cuda-include=<path/to/cuda/include> \\
    --with-cuda-libpath=<path/to/cuda/libraries>}

In addition to these we have added the following variables to help account for libraries being installed in different locations:\\
\texttt{--with-libcuda=<path/to/directory/containing/libcuda>}\\
\texttt{--with-libcudart=<path/to/directory/containing/libcudart}

\begin{small}
    \paragraph{Note:} If using the PGI compiler, you will need to add the
    following to your \texttt{CPPFLAGS} and \texttt{CFLAGS}.  You'll also need
    to use the \texttt{--enable-cuda=basic} configure option to build properly.
    See the example below.
    \begin{description}
        \item[Example:] \texttt{./configure --enable-cuda=basic
            CPPFLAGS="-D\_\_x86\_64 \\
            -D\_\_align\_\_\textbackslash(n\textbackslash)=\_\_attribute\_\_\textbackslash(\textbackslash(aligned\textbackslash(n\textbackslash)\textbackslash)\textbackslash) \\
            -D\_\_location\_\_\textbackslash(a\textbackslash)=\_\_annotate\_\_\textbackslash(a\textbackslash) \\
            -DCUDARTAPI="\\
            CFLAGS="-ta=tesla:nordc"}
    \end{description}
\end{small}

\subsection{Configuring a build to support running jobs across multiple
InfiniBand subnets}
\label{subsec:config-multi-subnet}

The support for running jobs across multiple subnets in MVAPICH2 can be enabled
at configure time as follows:

\CommandBox{\$ ./configure --enable-multi-subnet}{0.7}

MVAPICH2 relies on RDMA\_CM module to establish connections with peer processes.
The RDMA\_CM modules shipped some older versions of OFED (like OFED-1.5.4.1), do
not have the necessary support to enable communication across multiple subnets.
MVAPICH2 is capable of automatically detecting such OFED installations at
configure time. If the OFED installation present on the system does not support
running across multiple subnets, the configure step will detect this and exit
with an error message.

\subsection{Configuring a build for Shared-Memory-CH3}
\label{subsec:config-gen2-shm}

The default CH3 channel provides native support for shared memory communication
on stand alone multi-core nodes that are not equipped with InfiniBand adapters.
The steps to configure CH3 channel explicitly can be found in Section
~\ref{subsec:config-gen2}.  Dynamic Process Management (\ref{subsec:dpm}) is
currently not supported on stand-alone nodes without InfiniBand adapters.

\subsection{Configuring a build for OFA-IB-Nemesis}
\textcolor{red}{The Nemesis sub-channel for OFA-IB is now deprecated.} It can be built with:

\CommandBox{\$ ./configure --with-device=ch3:nemesis:ib}{0.7}

Both static and shared libraries are built by default. In order to build with static libraries only, configure as follows:

\CommandBox{\$ ./configure --with-device=ch3:nemesis:ib --disable-shared}{0.9}

To enable use of the TotalView debugger, the library needs to be configured
in the following manner:

\CommandBox{\$ ./configure --with-device=ch3:nemesis:ib \
                           --enable-g=dbg --enable-debuginfo}{0.9}

Configuration options for OFA-IB-Nemesis:

\begin{itemize}
	\item Configuring with Shared Libraries
		\begin{itemize}
			\item Default: Enabled
			\item Enable: \texttt{--enable-shared}
            \item Disable: \texttt{--disable-shared}
		\end{itemize}

	\item Configuring with TotalView support
		\begin{itemize}
			\item Default: Disabled
			\item Enable: \texttt{--enable-g=dbg \\
				--enable-debuginfo}
		\end{itemize}

        \item Path to IB Verbs
        \begin{itemize}
            \item Default: System Path
            \item Specify: \texttt{--with-ibverbs=<path>} or
            \newline \texttt{--with-ibverbs-include=<path>} and \texttt{--with-ibverbs-lib=<path>}
        \end{itemize}

	\item Registration Cache
	\begin{itemize}
		\item Default: enabled
		\item Disable: \texttt{--disable-registration-cache}
	\end{itemize}

	\item Header Caching
	\begin{itemize}
		\item Default: enabled
		\item Disable: \texttt{--disable-header-caching}
	\end{itemize}
	\item Support for 64K or greater number of cores
		\begin{itemize}
			\item Default: 64K or lower number of cores
			\item Enable: \texttt{--with-ch3-rank-bits=32}
		\end{itemize}

  \item Checkpoint/Restart
    \begin{itemize}
      \item Default: disabled
      \item Enable: \texttt{--enable-checkpointing} and \texttt{--with-hydra-ckpointlib=blcr}
      \item Require: Berkeley Lab Checkpoint/Restart (BLCR)
    \end{itemize}

    The Berkeley Lab Checkpoint/Restart (BLCR) installation is automatically detected if installed
in the standard location. To specify an alternative path to the BLCR installation, you can either use:\\
    \texttt{--with-blcr=<path/to/blcr/installation> } \\
    or \\
    \texttt{--with-blcr-include=<path/to/blcr/headers> \\ --with-blcr-libpath=<path/to/blcr/library>}

\end{itemize}


\subsection{Configuring a build for Intel TrueScale (PSM-CH3)}
\label{subsec:config-psm}

The TrueScale (PSM-CH3) interface needs to be built to use MVAPICH2 on Intel
TrueScale adapters. It can built with:

\CommandBox{\$ ./configure --with-device=ch3:psm}{0.7}

Both static and shared libraries are built by default. In order to build with static libraries only, configure as follows:

\CommandBox{\$ ./configure --with-device=ch3:psm --disable-shared}{0.9}

To enable use of the TotalView debugger, the library needs to be configured
in the following manner:

\CommandBox{\$ ./configure --with-device=ch3:psm \
                           --enable-g=dbg --enable-debuginfo}{0.9}

Configuration options for Intel TrueScale PSM channel:

\begin{itemize}
	\item Configuring with Shared Libraries
		\begin{itemize}
			\item Default: Enabled
			\item Enable: \texttt{--enable-shared}
            \item Disable: \texttt{--disable-shared}
		\end{itemize}

	\item Configuring with TotalView support
		\begin{itemize}
			\item Default: Disabled
			\item Enable: \texttt{--enable-g=dbg \
                                   --enable-debuginfo}
		\end{itemize}

    \item{Path to Intel TrueScale PSM header files}
    \begin{itemize}
        \item{Default: The systems search path for header files}
        \item{Specify: \texttt{--with-psm-include=path}}
    \end{itemize}
    
    \item{Path to Intel TrueScale PSM library}
    \begin{itemize}
        \item{Default: The systems search path for libraries}
        \item{Specify: \texttt{--with-psm-lib=path}}
        \end{itemize}

	\item Support for 64K or greater number of cores
		\begin{itemize}
			\item Default: 64K or lower number of cores
			\item Enable: \texttt{--with-ch3-rank-bits=32}
		\end{itemize}
       \end{itemize}

\subsection{Configuring a build for Intel Omni-Path (PSM2-CH3)}
\label{subsec:config-psm2}

The Omni-Path (PSM2-CH3) interface needs to be built to use MVAPICH2 on Intel
Omni-Path adapters. It can built with:

\CommandBox{\$ ./configure --with-device=ch3:psm}{0.7}

Both static and shared libraries are built by default. In order to build with static libraries only, configure as follows:

\CommandBox{\$ ./configure --with-device=ch3:psm --disable-shared}{0.9}

To enable use of the TotalView debugger, the library needs to be configured
in the following manner:

\CommandBox{\$ ./configure --with-device=ch3:psm \
                           --enable-g=dbg --enable-debuginfo}{0.9}

Configuration options for Intel Omni-Path PSM2 channel:

\begin{itemize}
	\item Configuring with Shared Libraries
		\begin{itemize}
			\item Default: Enabled
			\item Enable: \texttt{--enable-shared}
            \item Disable: \texttt{--disable-shared}
		\end{itemize}

	\item Configuring with TotalView support
		\begin{itemize}
			\item Default: Disabled
			\item Enable: \texttt{--enable-g=dbg \
                                   --enable-debuginfo}
		\end{itemize}

    \item{Path to Intel Omni-Path PSM2 header files}
    \begin{itemize}
        \item{Default: The systems search path for header files}
        \item{Specify: \texttt{--with-psm2-include=path}}
    \end{itemize}
    
    \item{Path to Intel Omni-Path PSM2 library}
    \begin{itemize}
        \item{Default: The systems search path for libraries}
        \item{Specify: \texttt{--with-psm2-lib=path}}
        \end{itemize}

	\item Support for 64K or greater number of cores
		\begin{itemize}
			\item Default: 64K or lower number of cores
			\item Enable: \texttt{--with-ch3-rank-bits=32}
		\end{itemize}
       \end{itemize}

\subsection{Configuring a build for TCP/IP-Nemesis}
\label{subsec:config-tcpip_nemesis}

The use of TCP/IP with Nemesis channel requires the following configuration:

\CommandBox{\$ ./configure --with-device=ch3:nemesis}{0.7}

Both static and shared libraries are built by default. In order to build with static libraries only, configure as follows:

\CommandBox{\$ ./configure --with-device=ch3:nemesis --disable-shared}{0.9}

To enable use of the TotalView debugger, the library needs to be configured
in the following manner:

\CommandBox{\$ ./configure --with-device=ch3:nemesis \
                           --enable-g=dbg --enable-debuginfo}{0.9}

Additional instructions for configuring with TCP/IP-Nemesis can be found in the MPICH
documentation available at:
~\href{http://www.mcs.anl.gov/research/projects/mpich2/documentation/index.php?s=docs}{http://www.mcs.anl.gov/research/projects/mpich2/documentation/index.php?s=docs}

\begin{itemize}
	\item Configuring with Shared Libraries
		\begin{itemize}
			\item Default: Enabled
			\item Enable: \texttt{--enable-shared}
            \item Disable: \texttt{--disable-shared}
		\end{itemize}

	\item Configuring with TotalView support
		\begin{itemize}
			\item Default: Disabled
			\item Enable: \texttt{--enable-g=dbg \\
                                   --enable-debuginfo}
		\end{itemize}
	\item Support for 64K or greater number of cores
		\begin{itemize}
			\item Default: 64K or lower number of cores
			\item Enable: \texttt{--with-ch3-rank-bits=32}
		\end{itemize}
\end{itemize}

\subsection{Configuring a build for TCP/IP-CH3}
\label{subsec:config-tcpip_ch3}

The use of TCP/IP requires the explicit selection of a TCP/IP enabled channel.
The recommended channel is TCP/IP Nemesis (described in Section
~\ref{subsec:config-tcpip_nemesis-and-ofa_nemesis}).  The alternative
ch3:sock channel can be selected by configuring with:

\CommandBox{\$ ./configure --with-device=ch3:sock}{0.7}

Both static and shared libraries are built by default. In order to build with static libraries only, configure as follows:

\CommandBox{\$ ./configure --with-device=ch3:sock --disable-shared}{0.9}

To enable use of the TotalView debugger, the library needs to be configured
in the following manner:

\CommandBox{\$ ./configure --with-device=ch3:sock \
                           --enable-g=dbg --enable-debuginfo}{0.9}

\begin{itemize}
	\item Configuring with Shared Libraries
		\begin{itemize}
			\item Default: Enabled
			\item Enable: \texttt{--enable-shared}
            \item Disable: \texttt{--disable-shared}
		\end{itemize}

	\item Configuring with TotalView support
		\begin{itemize}
			\item Default: Disabled
			\item Enable: \texttt{--enable-g=dbg \\
                                   --enable-debuginfo}
		\end{itemize}
	\item Support for 64K or greater number of cores
		\begin{itemize}
			\item Default: 64K or lower number of cores
			\item Enable: \texttt{--with-ch3-rank-bits=32}
		\end{itemize}
\end{itemize}

Additional instructions for configuring with TCP/IP can be found in the MPICH
documentation available at:

~\href{http://www.mpich.org/documentation/guides/}{http://www.mpich.org/documentation/guides/}

\subsection{Configuring a build for OFA-IB-Nemesis and TCP/IP Nemesis
(unified binary)}
\label{subsec:config-tcpip_nemesis-and-ofa_nemesis}

MVAPICH2 supports a unified binary for both OFA and TCP/IP communication
through the Nemesis interface.

In order to configure MVAPICH2 for unified binary support, perform the
following steps:

\CommandBox{\$ ./configure --with-device=ch3:nemesis:ib,tcp}{0.7}

You can use mpicc as usual to compile MPI applications. In order to
run your application on OFA:

\CommandBox{\$ mpiexec -f hosts ./a.out -n 2}{0.7}

To run your application on TCP/IP:

\CommandBox{\$ MPICH\_NEMESIS\_NETMOD=tcp mpiexec -f hosts
./osu\_latency -n 2}{0.9}

\subsection{Configuring a build for Shared-Memory-Nemesis}
\label{subsec:config-smp_nemesis}

The use of Nemesis shared memory channel requires the following
configuration.

\CommandBox{\$ ./configure --with-device=ch3:nemesis}{0.7}

Both static and shared libraries are built by default. In order to build with static libraries only, configure as follows:

\CommandBox{\$ ./configure --with-device=ch3:nemesis --disable-shared}{0.9}

To enable use of the TotalView debugger, the library needs to be configured
in the following manner:

\CommandBox{\$ ./configure --with-device=ch3:nemesis \
                           --enable-g=dbg --enable-debuginfo}{0.9}

Additional instructions for configuring with Shared-Memory-Nemesis can be found in the MPICH
documentation available at: \\
~\href{http://www.mcs.anl.gov/research/projects/mpich2/documentation/index.php?s=docs}{http://www.mcs.anl.gov/research/projects/mpich2/documentation/index.php?s=docs}

\begin{itemize}
	\item Configuring with Shared Libraries
		\begin{itemize}
			\item Default: Enabled
			\item Enable: \texttt{--enable-shared}
            \item Disable: \texttt{--disable-shared}
		\end{itemize}

	\item Configuring with TotalView support
		\begin{itemize}
			\item Default: Disabled
			\item Enable: \texttt{--enable-g=dbg \\
                                   --enable-debuginfo}
		\end{itemize}
\end{itemize}

\subsection{Configuration and Installation with Singularity}
\label{subsec:config-install-singularity}

MVAPICH2 can be configured and installed with Singularity in the following
manner. Note that the following prerequisites must be fulfilled before this
step.

\begin{itemize}
    \item{Singularity must be installed and operational}
    \item{Singularity image must be created as appropriate}
\end{itemize}

\paragraph{Sample Configuration and Installation with Singularity}
\begin{verbatim}
 # Clone the MVAPICH2 in current directory (on host)
 $ git@scm.mvapich.cse.ohio-state.edu:mvapich2.git
 $ cd mvapich2

 # Build MVAPICH2 in the working directory, 
 # using the tool chain within the container
 $ singularity exec Singularity-Centos-7.img ./autogen.sh
 $ singularity exec Singularity-Centos-7.img ./configure --prefix=/usr/local
 $ singularity exec Singularity-Centos-7.img make

 # Install MVAPICH2 into the container as root and writeable option
 $ sudo singularity exec -w Singularity-Centos-7.img make install
\end{verbatim}

\subsection{Installation with Spack}

MVAPICH2 can be configured and installed with Spack. For detailed instruction of
installing MVAPICH2 with Spack, please refer to Spack userguide in:\\
~\href{http://mvapich.cse.ohio-state.edu/userguide/userguide\_spack/}{http://mvapich.cse.ohio-state.edu/userguide/userguide\_spack/} 
