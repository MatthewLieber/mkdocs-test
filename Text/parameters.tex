\section{MVAPICH2 Parameters (CH3-Based Interfaces)}
\label{def:mvapich-parameters}

%\subsection{MV2\_3DTORUS\_SUPPORT}
%\label{def:mv2-3dtorus-support}
%\begin{itemize}
%    \item Class: Run time
%    \item Default: Disabled
%    \item Value Domain: $<$0, 1$>$
%    \item Applicable interface(s): OFA-IB-CH3, OFA-iWARP-CH3
%\end{itemize}
%When this environment variable is set, the MPI library will query the
%Subnet Manager for the correct Service Level to be used for a given path
%between a pair LIDs in a 3D torus network.

\subsection{MV2\_ALLREDUCE\_2LEVEL\_MSG} 
\label{def:mv2_allreduce_2level_msg}
\begin{itemize}
    \item Class: Run Time
    \item Default: 256K Bytes
    \item Applicable interface(s):  OFA-IB-CH3 and OFA-iWARP-CH3
\end{itemize}
This parameter can be used to determine the threshold for the 
2-level Allreduce algorithm. We now use the shared-memory-based 
algorithm for messages smaller than the \\
MV2\_SHMEM\_ALLREDUCE\_MSG threshold
(\ref{def:mv2-shmem-coll-allreduce-threshold}), 
the 2-level algorithm for medium sized messages up to the threshold 
defined by this parameter. We use the default point-to-point algorithms messages
larger than this threshold. 

\subsection{MV2\_CKPT\_AGGREGATION\_BUFPOOL\_SIZE}
\label{def:mv2_ckpt_aggregation_bufpool_size}
\begin{itemize}
    \item Class: Run Time
    \item Default: 8M
    \item Applicable interface(s):  OFA-IB-CH3
\end{itemize}
This parameter determines the size of the buffer pool reserved for use in checkpoint aggregation.
Note that this variable can be set with suffixes such as `K'/`k', `M'/`m' or `G'/`g'
to denote Kilobyte, Megabyte or Gigabyte respectively.

\subsection{MV2\_CKPT\_AGGREGATION\_CHUNK\_SIZE}
\label{def:mv2_ckpt_aggregation_chunk_size}
\begin{itemize}
    \item Class: Run Time
    \item Default: 1M
    \item Applicable interface(s):  OFA-IB-CH3
\end{itemize}
The checkpoint data that has been coalesced into the buffer pool, is written to the back-end file system,
with the value of this parameter as the chunk size.
Note that this variable can be set with suffixes such as `K'/`k', `M'/`m' or `G'/`g'
to denote Kilobyte, Megabyte or Gigabyte respectively.


\subsection{MV2\_CKPT\_FILE}
\label{def:mv2-ckpt-file}
\begin{itemize}
    \item Class: Run Time
    \item Default: /tmp/ckpt
    \item Applicable interface(s): OFA-IB-CH3
\end{itemize}
This parameter specifies the path and the base file name for
checkpoint files of MPI processes. The checkpoint files will be
named as \${MV2\_CKPT\_FILE.$<$number of checkpoint$>$.$<$process rank$>$},
for example, /tmp/ckpt.1.0 is the checkpoint file for process 0's
first checkpoint. To checkpoint on network-based file systems, user
just need to specify the path to it, such as
/mnt/pvfs2/my\_ckpt\_file.

\subsection{MV2\_CKPT\_INTERVAL}
\label{def:mv2-ckpt-interval}
\begin{itemize}
    \item Class: Run Time
    \item Default: 0
    \item Unit: minutes
    \item Applicable interface(s): OFA-IB-CH3
\end{itemize}
This parameter can be used to enable automatic checkpointing. To let
MPI job console automatically take checkpoints, this value needs to
be set to the desired checkpointing interval. A zero will disable
automatic checkpointing. Using automatic checkpointing, the
checkpoint file for the MPI job console will be named as
\${MV2\_CKPT\_FILE.$<$number of checkpoint$>$.auto}. Users need to use
this file for restart.

\subsection{MV2\_CKPT\_MAX\_SAVE\_CKPTS}
\label{def:mv2-max-save-ckpts}
\begin{itemize}
    \item Class: Run Time
    \item Default: 0
    \item Applicable interface(s): OFA-IB-CH3
\end{itemize}
This parameter is used to limit the number of checkpoints saved on
file system to save the file system space. When set to a positive
value N, only the last N checkpoints will be saved.

\subsection{MV2\_CKPT\_NO\_SYNC}
\label{def:mv2-ckpt-no-sync}
\begin{itemize}
    \item Class: Run Time
    \item Applicable interface(s): OFA-IB-CH3
\end{itemize}
When this parameter is set to any value, the checkpoints will not be
required to sync to disk. It can reduce the checkpointing delay in
many cases. But if users are using local file system, or any
parallel file system with local cache, to store the checkpoints, it
is recommended not to set this parameter because otherwise the
checkpoint files will be cached in local memory and will likely be
lost upon failure.

\subsection{MV2\_CKPT\_USE\_AGGREGATION}
\label{def:mv2-ckpt-use-aggregation}
\begin{itemize}
    \item Class: Run Time
    \item Default: 1 (if configured with Checkpoint Aggregation support)
    \item Applicable interface(s): OFA-IB-CH3
\end{itemize}
This parameter enables/disables Checkpoint aggregation scheme at run time.
It is set to '1'(enabled) by default, when the user enables Checkpoint/Restart 
functionality at configure time, or when the user explicitly configures
 MVAPICH2 with aggregation support. Please note that, to use aggregation support,
each node needs to be properly configured with FUSE library
(cf section~\ref{para:mpi-cr-aggr}).

\subsection{MV2\_DEBUG\_FT\_VERBOSE}
\label{def:mv2-debug-ft-verbose}
\begin{itemize}
    \item Class: Run Time
    \item Type: Null or positive integer
    \item Default: 0 (disabled)
\end{itemize}

This parameter enables/disables the debug output for Fault Tolerance features (Checkpoint/Restart and Migration).

Note: All debug output is disabled when MVAPICH2 is configured with the \texttt{--enable-fast=ndebug} option.


\subsection{MV2\_CM\_RECV\_BUFFERS}
\label{def:mv2-cm-recv-buffers}
\begin{itemize}
    \item Class: Run Time
    \item Default: 1024
    \item Applicable interface(s): OFA-IB-CH3
\end{itemize}
This defines the number of buffers used by connection manager to
establish new connections. These buffers are quite small and are
shared for all connections, so this value may be increased to 8192
for large clusters to avoid retries in case of packet drops.

\subsection{MV2\_CM\_SPIN\_COUNT}
\label{def:mv2-cm-spin-count}
\begin{itemize}
    \item Class: Run Time
    \item Default: 5000
    \item Applicable interface(s): OFA-IB-CH3
\end{itemize}
This is the number of the connection manager polls for new control
messages from UD channel for each interrupt. This may be increased
to reduce the interrupt overhead when many incoming control messages
from UD channel at the same time.


\subsection{MV2\_CM\_TIMEOUT}
\label{def:mv2-cm-timeout}
\begin{itemize}
    \item Class: Run Time
    \item Default: 500
    \item Unit: milliseconds
    \item Applicable interface(s): OFA-IB-CH3
\end{itemize}
This is the timeout value associated with connection management
messages via UD channel. Decreasing this value may lead to faster
retries but at the cost of generating duplicate messages.

\subsection{MV2\_CPU\_MAPPING}
\label{def:mv2-cpu-mapping}
\begin{itemize}
    \item Class: Run Time
    \item Default: NA
    \item Applicable interface(s):  OFA-IB-CH3, OFA-iWARP-CH3, PSM
\end{itemize}

This allows users to specify process to CPU (core) mapping. The detailed usage
of this parameter is described in Section~\ref{usage:mv2_cpu_mapping}. This 
parameter will not take effect if either \textit{MV2\_ENABLE\_AFFINITY} or \textit{MV2\_USE\_SHARED\_MEM}
run-time parameters are set to 0, or
if the library was configured with the ``--disable-hwloc'' option.
MV2\_CPU\_MAPPING is currently not supported on Solaris.

\subsection{MV2\_CPU\_BINDING\_POLICY}
\label{def:mv2-cpu-binding-policy}
\begin{itemize}
    \item Class: Run Time
    \item Default: hybrid 
    \item Applicable interface(s):  OFA-IB-CH3, OFA-iWARP-CH3, PSM
\end{itemize}

We have changed the default value of MV2\_CPU\_BINDING\_POLICY to ``hybrid''
along with MV2\_HYBRID\_BINDING\_POLICY=bunch. It is same as setting  
MV2\_CPU\_BINDING\_POLICY to bunch. However, it also works well for 
the systems with hyper-threading enabled or systems that have vendor specific 
core mappings.
This allows users to specify process to CPU (core) mapping with the CPU binding
policy. The detailed usage of this parameter is described in 
Section~\ref{usage:mv2_use_hwloc_cpu_binding}. This parameter will not take 
effect: if \\
\textit {MV2\_ENABLE\_AFFINITY} or \textit{MV2\_USE\_SHARED\_MEM} run-time parameters
are set to 0; or \\
\textit{MV2\_ENABLE\_AFFINITY} is set to 1 and \textit{MV2\_CPU\_MAPPING} is
set, or if the library was configured with the ``--disable-hwloc'' option.
The value of MV2\_CPU\_BINDING\_POLICY can be ``bunch'', ``scatter'', or
``hybrid''.  When this parameter takes effect and its value isn't set, ``bunch''
will be used as the default policy.


\subsection{MV2\_HYBRID\_BINDING\_POLICY}
\label{def:mv2-threads-binding-policy}
\begin{itemize}
    \item Class: Run Time
    \item Default: Linear 
    \item Applicable interface(s):  OFA-IB-CH3, OFA-iWARP-CH3, PSM
\end{itemize}

This allows users to specify binding policies for application thread in
MPI+Threads applications. The detailed usage of this parameter is described in 
Section~\ref{sec:advanced_thread_binding_policies}. This parameter will not take 
effect: if \\
\textit {MV2\_ENABLE\_AFFINITY} or \textit{MV2\_USE\_SHARED\_MEM} run-time parameters
are set to 0; or \\
\textit{MV2\_CPU\_BINDING\_POLICY} is set to ``bunch" or ``scatter", or \\
\textit{MV2\_ENABLE\_AFFINITY} is set to 1 and \textit{MV2\_CPU\_MAPPING} is
set, or if the library was configured with the ``--disable-hwloc'' option.
The value of MV2\_HYBRID\_BINDING\_POLICY can be ``linear'', ``compact'', ``bunch'', 
``scatter'', ``spread'', or ``numa''. 
When this parameter takes effect and its value isn't set, ``linear''
will be used as the default policy.

%This parameter is temporarily not exposed to the users.
%\subsection{MV2\_ENABLE\_LEASTLOAD}
%\label{def:mv2-enable-leastload}
%\begin{itemize}
%    \item Class: Run Time
%    \item Default: 0
%    \item Applicable interface(s): OFA-IB-CH3, OFA-iWARP-CH3, uDAPL-CH3 (Linux)
%\end{itemize}

%This allows users to specify \texttt {MV2\_CPU\_BINDING\_POLICY} 
%considering the 
%load on CPU (core). The detailed usage of this parameter is described in 
%Section~\ref{usage:mv2_use_hwloc_cpu_binding}. This parameter will not take 
%effect if \texttt{MV2\_CPU\_BINDING\_POLICY} is not set, or 
%\texttt {MV2\_CPU\_BINDING\_POLICY} is not set to 1, due 
%to its default value is 0.

\subsection{MV2\_CPU\_BINDING\_LEVEL}
\label{def:mv2-cpu-binding-level}
\begin{itemize}
    \item Class: Run Time
    \item Default: Core
    \item Applicable interface(s):  OFA-IB-CH3 and OFA-iWARP-CH3
\end{itemize}

This allows users to specify process to CPU (core) mapping at different
binding level. The detailed usage of this parameter is described in
Section~\ref{usage:mv2_use_hwloc_cpu_binding}. This parameter will not take
effect: if \textit {MV2\_ENABLE\_AFFINITY} or \textit{MV2\_USE\_SHARED\_MEM}
run-time parameters
are set to 0; or \texttt{MV2\_ENABLE\_AFFINITY} is set to 1 and
\texttt{MV2\_CPU\_MAPPING} is
set, or if the library was configured with the ``--disable-hwloc'' option.
The value of MV2\_CPU\_BINDING\_LEVEL can be ``core'', ``socket'', or
``numanode''. When this parameter takes effect and its value isn't set,
``core'' will be used as the default binding level.

\subsection{MV2\_SHOW\_HCA\_BINDING}
\label{def:show-hca-binding}
\begin{itemize}
    \item Class: Run time
    \item Default: 0 (disabled)
    \item Applicable interface(s):  OFA-IB-CH3, OFA-iWARP-CH3
    \item Possible values: 1 (Show only on node containing rank 0), 2 (Show on all nodes)
\end{itemize}

If set to 1, it shows the HCA binding of all processes on node where rank 0
exists.  If set to 2, it shows the HCA binding of all processes on all nodes.

\subsection{MV2\_DEFAULT\_MAX\_SEND\_WQE}
\label{def:max-send-wqe}

\begin{itemize}
        \item Class: Run time
        \item Default: 64
    \item Applicable interface(s):  OFA-IB-CH3 and OFA-iWARP-CH3
\end{itemize}

This specifies the maximum number of send WQEs on each QP. 
Please note that for OFA-IB-CH3 and OFA-iWARP-CH3, the default value of this parameter
will be 16 if the number of processes is larger than 256 for better
memory scalability.

\subsection{MV2\_DEFAULT\_MAX\_RECV\_WQE}
\label{def:max-recv-wqe}

\begin{itemize}
        \item Class: Run time
        \item Default: 128
    \item Applicable interface(s):  OFA-IB-CH3 and OFA-iWARP-CH3
\end{itemize}

This specifies the maximum number of receive WQEs on each QP (maximum
number of receives that can be posted on a single QP). 

\subsection{MV2\_DEFAULT\_MTU}
\label{def:rdma-default-mtu}

\begin{itemize}
    \item Class: Run time

    \item Default: OFA-IB-CH3: IBV\_MTU\_1024 for IB SDR cards and IBV\_MTU\_2048 for IB DDR and
		QDR cards.
    \item Applicable interface(s):  OFA-IB-CH3
\end{itemize}

The internal MTU size. For OFA-IB-CH3, this parameter should be a string instead
of an integer. Valid values are: \texttt{IBV\_MTU\_256}, \texttt{IBV\_MTU\_512},
\texttt{IBV\_MTU\_1024}, \texttt{IBV\_MTU\_2048}, \texttt{IBV\_MTU\_4096}.

\subsection{MV2\_DEFAULT\_PKEY}
\label{def:mv2-default-pkey}
\begin{itemize}
		\item Class: Run Time
		\item Applicable Interface(s): OFA-IB-CH3
\end{itemize}

Select the partition to be used for the job.

\subsection{MV2\_ENABLE\_AFFINITY}
\label{def:viadev_enable_affinity}

\begin{itemize}
    \item Class: Run time
    \item Default: 1
    \item Applicable interface(s): OFA-IB-CH3 and OFA-iWARP-CH3
\end{itemize}

Enable CPU affinity by setting MV2\_ENABLE\_AFFINITY to 1 or disable it by
setting \\
MV2\_ENABLE\_AFFINITY to 0. MV2\_ENABLE\_AFFINITY is currently not supported
on Solaris. CPU affinity is also not supported if MV2\_USE\_SHARED\_MEM is set to 0. 


\subsection{MV2\_GET\_FALLBACK\_THRESHOLD}
\begin{itemize}
        \item Class: Run time
        \item This threshold value needs to be set in bytes.
        \item This option is effective if we define ONE\_SIDED flag.
    \item Applicable interface(s): OFA-IB-CH3 and OFA-iWARP-CH3

\end{itemize}
This defines the threshold beyond which the MPI\_Get implementation is based on direct one sided RDMA operations.


\subsection{MV2\_IBA\_EAGER\_THRESHOLD}
\label{def:rdma-iba-eager-threshold}

\begin{itemize}
    \item Class: Run time

    \item Default: Host Channel Adapter (HCA) dependent (12 KB for ConnectX HCA's)
    \item Applicable interface(s): OFA-IB-CH3 and OFA-iWARP-CH3
\end{itemize}

This specifies the switch point between eager and rendezvous
protocol in MVAPICH2. For better performance, the value of 
MV2\_IBA\_EAGER\_THRESHOLD should be
set the same as MV2\_VBUF\_TOTAL\_SIZE.

\subsection{MV2\_IBA\_HCA}
\label{def:rdma-iba-hcas}

\begin{itemize}
    \item Class: Run time

    \item Default: Unset
    \item Applicable interface(s): OFA-IB-CH3 and OFA-iWARP-CH3
\end{itemize}

This specifies the HCA's to be used for performing network operations.

\subsection{MV2\_INITIAL\_PREPOST\_DEPTH}
\label{def:viadev-initial-prepost-depth}

\begin{itemize}
    \item Class: Run time

    \item Default: 10
    \item Applicable interface(s): OFA-IB-CH3 and OFA-iWARP-CH3
\end{itemize}

This defines the initial number of pre-posted receive buffers for each
connection. If communication happen for a particular connection, the
number of buffers will be increased to \\RDMA\_PREPOST\_DEPTH.

\subsection{MV2\_IWARP\_MULTIPLE\_CQ\_THRESHOLD}
\label{def:iwarp-multiple-cq-threshold}

\begin{itemize}
    \item Class: Run time

    \item Default: 32
    \item Applicable interface(s): OFA-iWARP-CH3
\end{itemize}

This defines the process size beyond which we use multiple completion queues for
iWARP interface.

\subsection{MV2\_KNOMIAL\_INTRA\_NODE\_FACTOR}
\label{def:mv2_knomial_intra_node_factor}
\begin{itemize}
    \item Class: Run time
    \item Default: 4
    \item Applicable interface(s): OFA-IB-CH3 and OFA-iWARP-CH3
\end{itemize}

This defines the degree of the knomial operation during the intra-node
knomial broadcast phase. 

\subsection{MV2\_KNOMIAL\_INTER\_NODE\_FACTOR}
\label{def:mv2_knomial_inter_node_factor}
\begin{itemize}
    \item Class: Run time
    \item Default: 4
    \item Applicable interface(s): OFA-IB-CH3 and OFA-iWARP-CH3
\end{itemize}

This defines the degree of the knomial operation during the inter-node
knomial broadcast phase. 

\subsection{MV2\_MAX\_INLINE\_SIZE}
\label{def:max-inline-size}
\begin{itemize}
    \item Class: Run time
    \item Default: Network card dependent (128 for most networks including InfiniBand)
    \item Applicable interface(s): OFA-IB-CH3, OFA-iWARP-CH3
\end{itemize}

This defines the maximum inline size for data transfer. Please note that the 
default value of this parameter will be 0 when the number of processes is larger than
256 to improve memory usage scalability.


\subsection{MV2\_MAX\_NUM\_WIN}
\label{def:max-num-win}
\begin{itemize}
    \item Class: Run time
    \item Default: 64
    \item Applicable interface(s): OFA-IB-CH3
\end{itemize}

Maximum number of RMA windows that can be created and active
concurrently. Typically this value is sufficient for most
applications. Increase this value to the number of windows your
application uses


\subsection{MV2\_NDREG\_ENTRIES}
\label{def:ndreg-entries}
\begin{itemize}
    \item Class: Run time
    \item Default: Represented by RDMA\_NDREG\_ENTRIES (value is 1100). Depends 
        on the number of processes.
    \item Applicable interface(s): OFA-IB-CH3 and OFA-iWARP-CH3
\end{itemize}

This defines the total number of buffers that can be stored in the
registration cache. It has no effect if MV2\_USE\_LAZY\_MEM\_UNREGISTER is
not set. A larger value will lead to less frequent lazy
de-registration.

\subsection{MV2\_NUM\_HCAS}
\label{def:num-hcas}
\begin{itemize}
    \item Class: Run time
    \item Default: 1
    \item Applicable interface(s): OFA-IB-CH3, OFA-iWARP-CH3
\end{itemize}
This parameter indicates number of InfiniBand adapters to be used for communication
on an end node.

\subsection{MV2\_NUM\_PORTS}
\label{def:num-ports}
\begin{itemize}
    \item Class: Run time
    \item Default: 1
    \item Applicable interface(s): OFA-IB-CH3, OFA-iWARP-CH3
\end{itemize}
This parameter indicates number of ports per InfiniBand adapter to be used for communication per adapter on an end node.

\subsection{MV2\_DEFAULT\_PORT}
\label{def:num-ports}
\begin{itemize}
    \item Class: Run time
    \item Default: none
    \item Applicable interface(s): OFA-IB-CH3, OFA-iWARP-CH3
\end{itemize}
This parameter is to select the specific HCA port on a active multi port InfiniBand adapter


\subsection{MV2\_NUM\_SA\_QUERY\_RETRIES}
\label{def:mv2_num_sa_query_retries}
\begin{itemize}
    \item Class: Run time
    \item Default: 20
	\item Applicable Interface(s): OFA-IB-CH3, OFA-iWARP-CH3
\end{itemize}
Number of times the MPI library will attempt to query the subnet to obtain
the path record information before giving up.

\subsection{MV2\_NUM\_QP\_PER\_PORT}
\label{def:num-qp-per-port}
\begin{itemize}
    \item Class: Run time
    \item Default: 1
    \item Applicable interface(s): OFA-IB-CH3, OFA-iWARP-CH3
\end{itemize}
This parameter indicates number of queue pairs
per port to be used for communication on an end node.
This is useful in the presence of multiple send/recv engines
available per port for data transfer.

\subsection{MV2\_RAIL\_SHARING\_POLICY}
\label{def:rail-sharing-policy}
\begin{itemize}
    \item Class: Run time
    \item Default: Rail Binding in round-robin
    \item Value Domain: USE\_FIRST, ROUND\_ROBIN, FIXED\_MAPPING
    \item Applicable interface(s): OFA-IB-CH3, OFA-iWARP-CH3
\end{itemize}
This specifies the policy that will be used to assign HCAs to each of the processes. In the 
previous versions of MVAPICH2 it was known as MV2\_SM\_SCHEDULING.

\subsection{MV2\_RAIL\_SHARING\_LARGE\_MSG\_THRESHOLD}
\label{def:rail-sharing-large-msg-threshold}
\begin{itemize}
    \item Class: Run time
    \item Default: 16K
    \item Applicable interface(s): OFA-IB-CH3, OFA-iWARP-CH3
\end{itemize}
This specifies the threshold for the message size beyond which striping will take place. 
In the previous versions of MVAPICH2 it was known as MV2\_STRIPING\_THRESHOLD

%\subsection{MV2\_PATH\_SL\_QUERY}
%\label{def:path-sl-query}
%\begin{itemize}
%    \item Class: Run time
%    \item Default: Disabled
%    \item Value Domain: $<$0, 1$>$
%    \item Applicable interface(s): OFA-IB-CH3, OFA-iWARP-CH3
%\end{itemize}
%When this environment variable is set, the MPI library will query the
%Subnet Manager for the correct Service Level to be used for a given path
%between a pair LIDs in the network.

\subsection{MV2\_PROCESS\_TO\_RAIL\_MAPPING}
\label{def:process-to-rail-mapping}
\begin{itemize}
    \item Class: Run time
    \item Default: NONE
    \item Value Domain: BUNCH, SCATTER, $<$CUSTOM LIST$>$
    \item Applicable interface(s): OFA-IB-CH3, OFA-iWARP-CH3
\end{itemize}
When MV2\_RAIL\_SHARING\_POLICY is set to the value ``FIXED\_MAPPING'' this variable 
decides the manner in which the HCAs will be mapped to the rails. The $<$CUSTOM LIST$>$ 
is colon(:) separated list with the HCA ranks specified. e.g. 0:1:1:0. This list must 
map equally to	the number of local processes on the nodes failing which, the default 
policy will be used. Similarly the number of processes on each node must be the
same. The detailed usage of this parameter is described in
Section~\ref{subsec:mpi-mr}.


\subsection{MV2\_RDMA\_FAST\_PATH\_BUF\_SIZE}
\label{def:rdma-fast-path-buf-size}

\begin{itemize}
    \item Class: Run time

    \item Default: Architecture dependent
    \item Applicable interface(s): OFA-IB-CH3 and OFA-iWARP-CH3

\end{itemize}

The size of the buffer used in RDMA fast path communication. 
This value will be ineffective if \\
MV2\_USE\_RDMA\_FAST\_PATH is not set

\subsection{MV2\_NUM\_RDMA\_BUFFER}
\label{def:num-rdma-buffer}

\begin{itemize}
    \item Class: Run time

    \item Default: Architecture dependent (32 for EM64T)
    \item Applicable interface(s): OFA-IB-CH3 and OFA-iWARP-CH3
\end{itemize}

The number of RDMA buffers used for the RDMA fast path. This \emph{fast
path} is used to reduce latency and overhead of small data and control
messages. This value will be ineffective if MV2\_USE\_RDMA\_FAST\_PATH is
not set. 

%\subsection{MV2\_NUM\_SLS}
%\label{def:num-rdma-sls}
%
%\begin{itemize}
%    \item Class: Run time
%
%    \item Default: 1
%    \item Applicable interface(s): OFA-IB-CH3
%\end{itemize}
%
%Specifies the number of Service Levels the user wants the MVAPICH2 library
%to use. Please refer to Section~\ref{subsec:mpi-qos} for more information
%on the QoS features supported by the MVAPICH2 library.

\subsection{MV2\_ON\_DEMAND\_THRESHOLD}
\label{def:mv2-on-demand-threshold}
\begin{itemize}
    \item Class: Run Time
    \item Default: 64 (OFA-IB-CH3), 16 (OFA-iWARP-CH3)
    \item Applicable interface(s): OFA-IB-CH3 and OFA-iWARP-CH3
\end{itemize}
This defines threshold for enabling on-demand connection management
scheme. When the size of the job is larger than the threshold value,
on-demand connection management will be used.

\subsection{MV2\_HOMOGENEOUS\_CLUSTER}
\label{def:mv2-homogeneous-cluster}
\begin{itemize}
    \item Class: Run Time
    \item Default: 0
    \item Applicable interface(s): OFA-IB-CH3 and OFA-iWARP-CH3
\end{itemize}
Set this parameter to 1 on homogeneous clusters to optimize the job start-up

\subsection{MV2\_PREPOST\_DEPTH}
\label{def:rdma-prepost-depth}

\begin{itemize}
    \item Class: Run time

    \item Default: 64
    \item Applicable interface(s): OFA-IB-CH3 and OFA-iWARP-CH3
\end{itemize}

This defines the number of buffers pre-posted for each connection to
handle send/receive operations.

\subsection{MV2\_PSM\_DEBUG}
\label{def:psm-debug}

\begin{itemize}
        \item Class: Run time (Debug)
        \item Default: 0
    \item Applicable interface: PSM
\end{itemize}

This parameter enables the dumping of run-time debug counters from the
MVAPICH2-PSM progress engine. Counters are dumped every PSM\_DUMP\_FREQUENCY
seconds.

\subsection{MV2\_PSM\_DUMP\_FREQUENCY}
\label{def:psm-dump}

\begin{itemize}
        \item Class: Run time (Debug)
        \item Default: 10 seconds
    \item Applicable interface: PSM
\end{itemize}

This parameters sets the frequency for dumping MVAPICH2-PSM debug counters.
Value takes effect only in PSM\_DEBUG is enabled.

\subsection{MV2\_PUT\_FALLBACK\_THRESHOLD}
\begin{itemize}
        \item Class: Run time
        \item This threshold value needs to be set in bytes.
        \item This option is effective if we define ONE\_SIDED flag.
    \item Applicable interface(s): OFA-IB-CH3 and OFA-iWARP-CH3

\end{itemize}
This defines the threshold beyond which the MPI\_Put implementation is based on
direct one sided RDMA operations.

\subsection{MV2\_RAIL\_SHARING\_LARGE\_MSG\_THRESHOLD}
\label{def:mv2-rail-sharing-large-msg-threshold}
\begin{itemize}
    \item Class: Run Time
    \item Default: 16 KB
    \item Applicable interface(s): OFA-IB-CH3, OFA-iWARP-CH3
\end{itemize}
This parameter specifies the message size above which we begin the stripe
the message across multiple rails (if present).

\subsection{MV2\_RDMA\_CM\_ARP\_TIMEOUT}
\label{def:mv2-rdma-cm-arp-timeout}
\begin{itemize}
    \item Class: Run Time
    \item Default: 2000 ms
    \item Applicable interface(s): OFA-IB-CH3, OFA-iWARP-CH3, OFA-RoCE-CH3
\end{itemize}
This parameter specifies the ARP timeout to be used by RDMA CM module.

\subsection{MV2\_RDMA\_CM\_MAX\_PORT}
\label{def:mv2-rdma-cm-max-port}
\begin{itemize}
    \item Class: Run Time
    \item Default: Unset
    \item Applicable interface(s): OFA-IB-CH3, OFA-iWARP-CH3, OFA-RoCE-CH3
\end{itemize}
This parameter specifies the upper limit of the port range to be used 
by the RDMA CM module when choosing the port on which it listens for 
connections.

\subsection{MV2\_RDMA\_CM\_MIN\_PORT}
\label{def:mv2-rdma-cm-min-port}
\begin{itemize}
    \item Class: Run Time
    \item Default: Unset
    \item Applicable interface(s): OFA-IB-CH3, OFA-iWARP-CH3, OFA-RoCE-CH3
\end{itemize}
This parameter specifies the lower limit of the port range to be used 
by the RDMA CM module when choosing the port on which it listens for 
connections.

\subsection{MV2\_REDUCE\_2LEVEL\_MSG} 
\label{def:mv2_reduce_2level_msg}
\begin{itemize}
    \item Class: Run Time
    \item Default: 32K Bytes. 
    \item Applicable interface(s): OFA-IB-CH3 and OFA-iWARP-CH3
\end{itemize}
This parameter can be used to determine the threshold for the 
2-level reduce algorithm. We now use the shared-memory-based 
algorithm for messages smaller than the MV2\_SHMEM\_REDUCE\_MSG
(\ref{def:mv2-shmem-coll-reduce-threshold}), 
the 2-level algorithm for medium sized messages up to the threshold 
defined by this parameter. We use the default point-to-point algorithms messages
larger than this threshold. 

\subsection{MV2\_RNDV\_PROTOCOL}
\label{def:mv2_rndv_protocol}
\begin{itemize}
    \item Class: Run time
    \item Default: RPUT
    \item Applicable interface(s): OFA-IB-CH3, OFA-iWARP-CH3
\end{itemize}
The value of this variable can be set to choose different Rendezvous
protocols. RPUT (default RDMA-Write) RGET (RDMA Read based), R3
(send/recv based).

\subsection{MV2\_R3\_THRESHOLD}
\label{def:mv2_r3_threshold}
\begin{itemize}
    \item Class: Run time
    \item Default: MV2\_IBA\_EAGER\_THRESHOLD
    \item Applicable interface(s): OFA-IB-CH3, OFA-iWARP-CH3
\end{itemize}

The value of this variable controls what message sizes go over the 
R3 rendezvous protocol. Messages above this message size use
MV2\_RNDV\_PROTOCOL. 

\subsection{MV2\_R3\_NOCACHE\_THRESHOLD}
\begin{itemize}
    \item Class: Run time
    \item Default: 32768
    \item Applicable interface(s): OFA-IB-CH3, OFA-iWARP-CH3
\end{itemize}

The value of this variable controls what message sizes go over the 
R3 rendezvous protocol when the registration cache is disabled (MV2\_USE\_LAZY\_MEM\_UNREGISTER=0). Messages above this message size use
MV2\_RNDV\_PROTOCOL. 

\subsection{MV2\_SHMEM\_ALLREDUCE\_MSG}
\label{def:mv2-shmem-coll-allreduce-threshold}
\begin{itemize}
    \item Class: Run Time
    \item Default: 1 $\ll$ 15
    \item Applicable interface(s): OFA-IB-CH3, OFA-iWARP-CH3
\end{itemize}

The SHMEM AllReduce is used for messages less than this threshold.

\subsection{MV2\_SHMEM\_BCAST\_LEADERS}
\label{def:mv2-shmem-bcast-leaders}

\begin{itemize}
        \item Class: Run time
        \item Default: 4096
\end{itemize}

The number of leader processes that will take part in the SHMEM
broadcast operation. Must be greater than the number of nodes in the
job.

\subsection{MV2\_SHMEM\_BCAST\_MSG}
\label{def:mv2-shmem-coll-bcast-threshold}
\begin{itemize}
    \item Class: Run Time
    \item Default: 1 $\ll$ 20
    \item Applicable interface(s): OFA-IB-CH3, OFA-iWARP-CH3
\end{itemize}
The SHMEM bcast is used for messages less than this threshold.

\subsection{MV2\_SHMEM\_COLL\_MAX\_MSG\_SIZE}
\label{def:shmem-coll-max-msg-size}
\begin{itemize}
    \item Class: Run Time
    \item Applicable interface(s): OFA-IB-CH3, OFA-iWARP-CH3
\end{itemize}
This parameter can be used to select the max buffer size of message
for shared memory collectives.

\subsection{MV2\_SHMEM\_COLL\_NUM\_COMM}
\label{def:shmem-coll-num-comm}
\begin{itemize}
    \item Class: Run Time
    \item Applicable interface(s): OFA-IB-CH3, OFA-iWARP-CH3
\end{itemize}
This parameter can be used to select the number of communicators
using shared memory collectives.

\subsection{MV2\_SHMEM\_DIR}
\label{def:shmem-dir}
\begin{itemize}
    \item Class: Run Time
    \item Applicable interface(s): OFA-IB-CH3 and OFA-iWARP-CH3
    \item Default: /dev/shm for Linux and /tmp for Solaris
\end{itemize}
This parameter can be used to specify the path to the shared memory
files for intra-node communication.

\subsection{MV2\_SHMEM\_REDUCE\_MSG}
\label{def:mv2-shmem-coll-reduce-threshold}
\begin{itemize}
    \item Class: Run Time
    \item Default: 1 $\ll$ 13
    \item Applicable interface(s): OFA-IB-CH3, OFA-iWARP-CH3
\end{itemize}
The SHMEM reduce is used for messages less than this threshold.

\subsection{MV2\_SM\_SCHEDULING}
\label{def:mv2-sm-scheduling}
\begin{itemize}
    \item Class: Run Time
    \item Default: USE\_FIRST (Options: ROUND\_ROBIN)
    \item Applicable interface(s): OFA-IB-CH3, OFA-iWARP-CH3
\end{itemize}

\subsection{MV2\_SMP\_USE\_LIMIC2}
\label{def:mv2-smp-limic2}
\begin{itemize}
    \item Class: Run Time
    \item Default: On if configured with --with-limic2
    \item Applicable interface(s): OFA-IB-CH3 and OFA-iWARP-CH3
\end{itemize}

This parameter enables/disables LiMIC2 at run time. It does
not take effect if MVAPICH2 is not configured with --with-limic2.

\subsection{MV2\_SMP\_USE\_CMA}
\label{def:mv2-smp-cma}
\begin{itemize}
    \item Class: Run Time
    \item Default: On unless configured with --without-cma
    \item Applicable interface(s): OFA-IB-CH3 and OFA-iWARP-CH3
\end{itemize}

This parameter enables/disables CMA based intra-node communication at run time. 
It does not take effect if MVAPICH2 is configured with --without-cma.
When --with-limic2 is included in the configure flags, LiMIC2 is used in
preference over CMA. Please set MV2\_SMP\_USE\_LIMIC2 to 0 in order to choose
CMA if MVAPICH2 is configured with --with-limic2.

\subsection{MV2\_SRQ\_LIMIT}
\label{def:viadev-srq-limit}

\begin{itemize}
    \item Class: Run Time
    \item Default: 30
    \item Applicable interface(s): OFA-IB-CH3, OFA-iWARP-CH3
\end{itemize}

This is the low water-mark limit for the
Shared Receive Queue. If the
number of available work entries on the
SRQ drops below this limit, the
flow control will be activated.


\subsection{MV2\_SRQ\_MAX\_SIZE}
\label{def:viadev-srq-max-size}

\begin{itemize}
    \item Class: Run Time
    \item Default: 4096
    \item Applicable interface(s): OFA-IB-CH3, OFA-iWARP-CH3
\end{itemize}

This is the maximum number of work requests allowed on the Shared Receive Queue. 
Upon receiving a SRQ limit event, the current value of MV2\_SRQ\_SIZE will be doubled 
or moved to the maximum of MV2\_SRQ\_MAX\_SIZE, whichever is smaller

\subsection{MV2\_SRQ\_SIZE}
\label{def:viadev-srq-size}

\begin{itemize}
    \item Class: Run Time
    \item Default: 256
    \item Applicable interface(s): OFA-IB-CH3, OFA-iWARP-CH3
\end{itemize}

This is the initial number of work
requests posted to the Shared
Receive Queue. 

\subsection{MV2\_STRIPING\_THRESHOLD}
\label{def:viadev-striping-threshold}

\begin{itemize}
    \item Class: Run Time
    \item Default: 8192
    \item Applicable interface(s): OFA-IB-CH3, OFA-iWARP-CH3
\end{itemize}

This parameter specifies the message size above which we begin the stripe the
message across multiple rails (if present).

\subsection{MV2\_SUPPORT\_DPM}
\label{def:support-dpm}

\begin{itemize}
	\item Class: Run time
	\item Default: 0 (disabled)
	\item Applicable interface: OFA-IB-CH3
\end{itemize}

This option enables the dynamic process management interface and on-demand connection
management.

\subsection{MV2\_USE\_APM}
\label{def:mv2-use-apm}
\begin{itemize}
	\item Class: Run Time
	\item Applicable interface(s): OFA-IB-CH3
\end{itemize}

This parameter is used for recovery from network faults using Automatic
Path Migration. This functionality is beneficial in the presence of
multiple paths in the network, which can be enabled by using lmc
mechanism. 

\subsection{MV2\_USE\_APM\_TEST}
\label{def:mv2-use-apm-test}
\begin{itemize}
	\item Class: Run Time
	\item Applicable interface(s): OFA-IB-CH3
\end{itemize}

This parameter is used for testing the Automatic Path Migration
functionality. It periodically moves the alternate path as the primary
path of communication and re-loads another alternate path.


\subsection{MV2\_USE\_BLOCKING}
\begin{itemize}
    \item Class: Run time
    \item Default: 0
    \item Applicable interface(s): OFA-IB-CH3
\end{itemize}
Setting this parameter enables MVAPICH2 to use blocking mode progress.
MPI applications do not take up any CPU when they are waiting for
incoming messages.


\subsection{MV2\_USE\_COALESCE}
\begin{itemize}
    \item Class: Run time
    \item Default: unset
    \item Applicable interface(s): OFA-IB-CH3, OFA-iWARP-CH3
\end{itemize}
Setting this parameter enables message coalescing to increase small
message throughput 

\subsection{MV2\_USE\_DIRECT\_GATHER}
\label{def:mv2_use_direct_gather}
\begin{itemize}
    \item Class: Run time
    \item Default: set
    \item Applicable interface(s): OFA-IB-CH3 and OFA-iWARP-CH3
\end{itemize}
Use the ``Direct'' algorithm for the MPI\_Gather operation. If this parameter is 
set to 0 at run-time, the ``Direct'' algorithm will not be invoked. 


\subsection{MV2\_USE\_DIRECT\_SCATTER}
\label{def:mv2_use_direct_scatter}
\begin{itemize}
    \item Class: Run time
    \item Default: set
    \item Applicable interface(s): OFA-IB-CH3 and OFA-iWARP-CH3
\end{itemize}
Use the ``Direct'' algorithm for the MPI\_Scatter operation. If this parameter is 
set to 0 at run-time, the ``Direct'' algorithm will not be invoked. 

\subsection{MV2\_USE\_HSAM}
\label{def:mv2-use-hsam}
\begin{itemize}
	\item Class: Run Time
	\item Applicable interface(s): OFA-IB-CH3
\end{itemize}

This parameter is used for utilizing hot-spot avoidance with InfiniBand
clusters. To leverage this functionality, the subnet should be
configured with lmc greater than zero. Please refer to
section~\ref{def:mv2-hsam} for detailed information.

\subsection{MV2\_USE\_IWARP\_MODE}
\label{def:mv2-enable-iwarp-mode}
\begin{itemize}
    \item Class: Run Time
    \item Default: unset
    \item Applicable interface(s): OFA-IB-CH3, OFA-iWARP-CH3
\end{itemize}
This parameter enables the library to run in iWARP mode. 

\subsection{MV2\_USE\_LAZY\_MEM\_UNREGISTER}
\label{def:mv2_use_lazy_mem_unregister}
\begin{itemize}
    \item Class: Run time
    \item Default: set
    \item Applicable interface(s): OFA-IB-CH3 and OFA-iWARP-CH3
\end{itemize}
Setting this parameter enables MVAPICH2 to use memory registration cache.

%\subsection{MV2\_USE\_QOS}
%\label{def:mv2_use_qos}
%\begin{itemize}
%    \item Class: Run time
%    \item Default: Un set
%    \item Applicable interface(s): OFA-IB-CH3
%\end{itemize}
%
%Setting this parameter enables MVAPICH2 to use InfiniBand's Quality of
%Service features. Please refer to Section~\ref{subsec:mpi-qos} for more
%information on the QoS features supported by the MVAPICH2 library.

\subsection{MV2\_USE\_RoCE}
\label{def:mv2-use-roce}
\begin{itemize}
    \item Class: Run Time
    \item Default: Un Set
    \item Applicable interface(s): OFA-IB-CH3
\end{itemize}
This parameter enables the use of RDMA over Ethernet for MPI communication.
The underlying HCA and network must support this feature.

\subsection{MV2\_DEFAULT\_GID\_INDEX}
\label{def:mv2-gid-index}
\begin{itemize}
    \item Class: Run Time
    \item Default: 0
    \item Applicable interface(s): OFA-IB-CH3
\end{itemize}
In RoCE mode, this parameter allows to choose non-default GID index in loss-less
ethernet setup using VLANs

\subsection{MV2\_USE\_RDMA\_CM}
\label{def:mv2-use-rdma-cm}
\begin{itemize}
    \item Class: Run Time
    \item Default: Network Dependent (set for OFA-iWARP-CH3 and unset for OFA-IB-CH3/OFA-RoCE-CH3)
    \item Applicable interface(s): OFA-IB-CH3, OFA-iWARP-CH3, OFA-RoCE-CH3
\end{itemize}
This parameter enables the use of RDMA CM for establishing the
connections. 

\subsection{MV2\_RDMA\_CM\_MULTI\_SUBNET\_SUPPORT}
\label{def:mv2-rdma-cm-multi-subnet-support}
\begin{itemize}
    \item Class: Run Time
    \item Default: Unset
    \item Applicable interface(s): OFA-IB-CH3, OFA-iWARP-CH3
\end{itemize}
This parameter allows MPI jobs to be run across multiple subnets interconnected
by InfiniBand routers. Note that this requires RDMA\_CM support to be enabled at
configure time and runtime. Note that, RDMA\_CM support is enabled by default at
configure time. At runtime, the MV2\_USE\_RDMA\_CM environment variable
described in Section~\ref{def:mv2-use-rdma-cm} must be set to 1.

\subsection{MV2\_RDMA\_CM\_CONF\_FILE\_PATH}
\label{def:mv2-rdma-cm-conf-file-path}
\begin{itemize}
    \item Class: Run Time
    \item Default: Network Dependent (set for OFA-iWARP-CH3 and unset for OFA-IB-CH3/OFA-RoCE-CH3)
    \item Applicable interface(s): OFA-IB-CH3, OFA-iWARP-CH3, OFA-RoCE-CH3
\end{itemize}
This parameter is to specify the path to mv2.conf file. If this is not given, then it searches
in the default location /etc/mv2.conf


\subsection{MV2\_USE\_RDMA\_FAST\_PATH}
\label{def:disable-rfp}
\begin{itemize}
    \item Class: Run time
    \item Default: set
    \item Applicable interface(s): OFA-IB-CH3 and OFA-iWARP-CH3
\end{itemize}
Setting this parameter enables MVAPICH2 to use adaptive RDMA fast path features
for OFA-IB-CH3 interface.



\subsection{MV2\_USE\_RDMA\_ONE\_SIDED}
\begin{itemize}
    \item Class: Run time
    \item Default: set
    \item Applicable interface(s): OFA-IB-CH3 and OFA-iWARP-CH3
\end{itemize}
Setting this parameter allows MVAPICH2 to use optimized one sided implementation
based RDMA operations.


\subsection{MV2\_USE\_RING\_STARTUP}
\begin{itemize}
    \item Class: Run time
    \item Default: set
    \item Applicable interface(s): OFA-IB-CH3
\end{itemize}
Setting this parameter enables MVAPICH2 to use ring based start up. 



\subsection{MV2\_USE\_SHARED\_MEM}
\label{def:use-shared-mem}
\begin{itemize}
    \item Class: Run time
    \item Default: set 
    \item Applicable interface(s): OFA-IB-CH3 and OFA-iWARP-CH3
\end{itemize}

Use shared memory for intra-node communication.

\subsection{MV2\_USE\_SHMEM\_ALLREDUCE}
\label{def:mv2-use-shmem-allreduce}
\begin{itemize}
		\item Class: Run Time
    \item Applicable interface(s): OFA-IB-CH3 and OFA-iWARP-CH3
\end{itemize}
This parameter can be used to turn off shared memory based
MPI\_Allreduce for OFA-IB-CH3 over IBA by setting this to 0.


\subsection{MV2\_USE\_SHMEM\_BARRIER}
\label{def:mv2-use-shmem-barrier}
\begin{itemize}
		\item Class: Run Time
    \item Applicable interface(s): OFA-IB-CH3 and OFA-iWARP-CH3
\end{itemize}
This parameter can be used to turn off shared memory based
MPI\_Barrier for OFA-IB-CH3 over IBA by setting this to 0.

\subsection{MV2\_USE\_SHMEM\_BCAST}
\label{def:mv2-use-shmem-bcast}
\begin{itemize}
                \item Class: Run Time
    \item Applicable interface(s): OFA-IB-CH3 and OFA-iWARP-CH3
\end{itemize}
This parameter can be used to turn off shared memory based
MPI\_Bcast for OFA-IB-CH3 over IBA by setting this to 0.

\subsection{MV2\_USE\_SHMEM\_COLL}
\label{def:mv2-use-shmem-coll}
\begin{itemize}
    \item Class: Run time
    \item Default: set 
    \item Applicable interface(s): OFA-IB-CH3 and OFA-iWARP-CH3
\end{itemize}

Use shared memory for collective communication. Set this to 0 for
disabling shared memory collectives.


\subsection{MV2\_USE\_SHMEM\_REDUCE}
\label{def:mv2-use-shmem-reduce}
\begin{itemize}
		\item Class: Run Time
    \item Applicable interface(s): OFA-IB-CH3 and OFA-iWARP-CH3
\end{itemize}
This parameter can be used to turn off shared memory based
MPI\_Reduce for OFA-IB-CH3 over IBA by setting this to 0.


\subsection{MV2\_USE\_SRQ}
\begin{itemize}
    \item Class: Run time
    \item Default: set
    \item Applicable interface(s): OFA-IB-CH3, OFA-iWARP-CH3
\end{itemize}
Setting this parameter enables MVAPICH2 to use shared receive queue.

\subsection{MV2\_GATHER\_SWITCH\_PT}
\label{def:mv2_gather_switch_pt}
\begin{itemize}
    \item Class: Run time 
    \item Default: set
    \item Applicable interface(s): OFA-IB-CH3 and OFA-iWARP-CH3
\end{itemize}
We use different algorithms depending on the system size. For small system sizes (up to 386 cores), we use 
the ``2-level'' algorithm following by the ``Direct'' algorithm. 
For medium system sizes (up to 1k), we use ``Binomial'' algorithm following by the ``Direct'' algorithm. 
Users can set the switching point between algorithms using the run-time parameter MV2\_GATHER\_SWITCH\_PT.

\subsection{MV2\_SCATTER\_SMALL\_MSG}
\label{def:mv2_scatter_small_msg}
\begin{itemize}
    \item Class: Run time
    \item Default: set
    \item Applicable interface(s): OFA-IB-CH3 and OFA-iWARP-CH3
\end{itemize}
When the system size is lower than 512 cores, we use the ``Binomial'' algorithm for small message sizes.
MV2\_SCATTER\_SMALL\_MSG allows the users to set the threshold for small messages.
                                                           
\subsection{MV2\_SCATTER\_MEDIUM\_MSG}
\label{def:mv2_scatter_medium_msg}
\begin{itemize}
    \item Class: Run time
    \item Default: set
    \item Applicable interface(s): OFA-IB-CH3 and OFA-iWARP-CH3
\end{itemize}
When the system size is lower than 512 cores, we use the ``2-level'' algorithm for medium message sizes.
MV2\_SCATTER\_MEDIUM\_MSG allows the users to set the threshold for medium messages.

\subsection{MV2\_USE\_TWO\_LEVEL\_GATHER}
\label{def:mv2_use_two_level_gather}
\begin{itemize}
    \item Class: Run time
    \item Default: set
    \item Applicable interface(s): OFA-IB-CH3 and OFA-iWARP-CH3
\end{itemize}
Use the two-level multi-core-aware algorithm for the MPI\_Gather operation. 
If this parameter is set to 0 at run-time, the two-level algorithm will not be invoked. 


\subsection{MV2\_USE\_TWO\_LEVEL\_SCATTER}
\label{def:mv2_use_two_level_scatter}
\begin{itemize}
    \item Class: Run time
    \item Default: set
    \item Applicable interface(s): OFA-IB-CH3 and OFA-iWARP-CH3
\end{itemize}
Use the two-level multi-core-aware algorithm for the MPI\_Scatter operation. 
If this parameter is set to 0 at run-time, the two-level algorithm will not be invoked. 

\subsection{MV2\_USE\_XRC}
\label{def:mv2_use_xrc}
\begin{itemize}
    \item Class: Run time
    \item Default: 0
	\item Applicable Interface(s): OFA-IB-CH3
\end{itemize}

Use the XRC InfiniBand transport available since Mellanox ConnectX adapters.
This features requires OFED version later than 1.3. It also automatically
enables SRQ and ON-DEMAND connection management. Note that the MVAPICH2
library needs to have been configured with --enable-xrc=yes to use this 
feature.

\subsection{MV2\_VBUF\_POOL\_SIZE}
\label{def:rdma-vbuf-pool-size}

\begin{itemize}
    \item Class: Run time

    \item Default: 512
    \item Applicable interface(s): OFA-IB-CH3, OFA-iWARP-CH3
\end{itemize}

The number of vbufs in the initial pool. This pool is shared among all
the connections.


\subsection{MV2\_VBUF\_SECONDARY\_POOL\_SIZE}
\label{def:rdma-vbuf-secondary-pool-size}

\begin{itemize}
    \item Class: Run time
    \item Default: 256
    \item Applicable interface(s): OFA-IB-CH3, OFA-iWARP-CH3
\end{itemize}

The number of vbufs allocated each time when the global pool is running
out in the initial pool. This is also shared among all the connections.

\subsection{MV2\_VBUF\_TOTAL\_SIZE}
\label{def:vbuf-total-size}

\begin{itemize}
    \item Class: Run time

    \item Default: Host Channel Adapter (HCA) dependent (12 KB for ConnectX HCA's)
    \item Applicable interface(s): OFA-IB-CH3, OFA-iWARP-CH3
\end{itemize}

The size of each \texttt{vbuf}, the basic communication buffer of MVAPICH2.
For better performance, the value of MV2\_IBA\_EAGER\_THRESHOLD should be
set the same as MV2\_VBUF\_TOTAL\_SIZE.

%\subsection{\_AFFINITY\_}
%\label{def:_affinity_}
%
%\begin{itemize}
%    \item Class: Compile time
%    \item Default: Defined by default on Opteron machines when \_SMP\_
%    is defined. Not defined by default on other platforms or when
%    \_SMP\_ is not defined.
%    \item Applicable interface(s): OFA-IB-CH3, OFA-iWARP-CH3, uDAPL-CH3
%\end{itemize}

%Use CPU affinity to improve data locality on NUMA based platforms.
%CPU affinity can be enabled/disabled by the run time parameter
%MV2\_ENABLE\_AFFINITY when \_AFFINITY\_ is defined.


\subsection{MV2\_SMP\_EAGERSIZE}
\label{def:smp-eagersize}

\begin{itemize}
    \item Class: Run time
    \item Default: Architecture dependent
    \item Applicable interface(s): OFA-IB-CH3 and OFA-iWARP-CH3
\end{itemize}

This parameter defines the
switch point from Eager protocol to Rendezvous protocol for intra-node
communication. Note that this variable can be set with suffixes such as `K'/`k', `M'/`m' or `G'/`g' 
to denote Kilobyte, Megabyte or Gigabyte respectively.


\subsection{MV2\_SMP\_QUEUE\_LENGTH}
\label{def:smp-queue-length}

\begin{itemize}
    \item Class: Run time

    \item Default: Architecture dependent
    \item Applicable interface(s): OFA-IB-CH3 and OFA-iWARP-CH3
\end{itemize}

This parameter defines the size
of shared buffer between every two processes on the same node for transferring
messages smaller than or equal to MV2\_SMP\_EAGERSIZE. 
Note that this variable can be set with suffixes such as `K'/`k', `M'/`m' or `G'/`g'
to denote Kilobyte, Megabyte or Gigabyte respectively.

\subsection{MV2\_SMP\_NUM\_SEND\_BUFFER}
\label{def:smp-num-send-buffer}

\begin{itemize}
        \item Class: Run time
        \item Default: Architecture dependent
    \item Applicable interface(s): OFA-IB-CH3 and OFA-iWARP-CH3
\end{itemize}

This parameter defines the
number of internal send buffers for sending intra-node messages larger
than MV2\_SMP\_EAGERSIZE.


\subsection{MV2\_SMP\_SEND\_BUF\_SIZE}
\label{def:smp-send-buf-size}

\begin{itemize}
        \item Class: Run time
        \item Default: Architecture dependent
    \item Applicable interface(s): OFA-IB-CH3 and OFA-iWARP-CH3
\end{itemize}

This parameter defines the
packet size when sending intra-node messages larger than \\
MV2\_SMP\_EAGERSIZE.


\subsection{MV2\_USE\_HUGEPAGES}
\label{def:use-hugepage}

\begin{itemize}
        \item Class: Run time
        \item Default: 1
        \item Applicable interface(s): OFA-IB-CH3
\end{itemize}
Set this to 0, to not use any HugePages.

\subsection{MV2\_HYBRID\_ENABLE\_THRESHOLD}
\label{def:ud-hybrid-threshold}

\begin{itemize}
        \item Class: Run time
        \item Default: 512
        \item Applicable interface(s): OFA-IB-CH3
\end{itemize}
This defines the threshold for enabling Hybrid communication
using UD and RC/XRC. When the size of the job is greater 
than or equal to the threshold value, Hybrid mode will be
enabled. Otherwise, it uses default RC/XRC connections for
communication.

\subsection{MV2\_HYBRID\_MAX\_RC\_CONN}
\label{def:max-rc-conn}

\begin{itemize}
        \item Class: Run time
        \item Default: 64
        \item Applicable interface(s): OFA-IB-CH3
\end{itemize}
Maximum number of RC or XRC connections created per process. This limits the 
amount of connection memory and prevents HCA QP cache thrashing.

\subsection{MV2\_UD\_PROGRESS\_TIMEOUT}
\label{def:ud-progress-timeout}

\begin{itemize}
        \item Class: Run time
        \item Default: System size dependent.
        \item Applicable interface(s): OFA-IB-CH3
\end{itemize}

Time (usec) until ACK status is checked (and ACKs are sent if needed). 
To avoid unnecessary retries, set this value less than MV2\_UD\_RETRY\_TIMEOUT.
It is recommended to set this to 1/10 of \\
MV2\_UD\_RETRY\_TIMEOUT.


\subsection{MV2\_UD\_RETRY\_TIMEOUT}
\label{def:ud-rettry-timeout}

\begin{itemize}
        \item Class: Run time
        \item Default: System size dependent.
        \item Applicable interface(s): OFA-IB-CH3
\end{itemize}

Time (usec) after which an unacknowledged message will be retried 

\subsection{MV2\_UD\_RETRY\_COUNT}
\label{def:ud-retry-count}

\begin{itemize}
        \item Class: Run time
        \item Default: System size dependent.
        \item Applicable interface(s): OFA-IB-CH3
\end{itemize}
Number of retries of a message before the job is aborted. This is needed in case of HCA fails. 

\subsection{MV2\_USE\_UD\_HYBRID}
\label{def:use-ud-hybrid}

\begin{itemize}
        \item Class: Run time
        \item Default: 1
        \item Applicable interface(s): OFA-IB-CH3
\end{itemize}
Set this to Zero, to disable UD transport in hybrid configuration mode.

\subsection{MV2\_USE\_ONLY\_UD}
\label{def:use-only-ud}

\begin{itemize}
        \item Class: Run time
        \item Default: 0
        \item Applicable interface(s): OFA-IB-CH3
\end{itemize}
Set this to 1, to enable only UD transport in hybrid configuration mode.
It will not use any RC/XRC connections in this mode.

\subsection{MV2\_USE\_UD\_ZCOPY}
\label{def:ud-zcopy}

\begin{itemize}
        \item Class: Run time
        \item Default: 1
        \item Applicable interface(s): OFA-IB-CH3
\end{itemize}
Whether or not to use the zero-copy transfer mechanism to transfer large messages on UD transport.

\subsection{MV2\_USE\_LIMIC\_GATHER}
\label{def:use-limic-gather}

\begin{itemize}
        \item Class: Run time
        \item Default: 0
    \item Applicable interface(s): OFA-IB-CH3, PSM
\end{itemize}
If this flag is set to 1, we will use intra-node Zero-Copy MPI\_Gather designs, when 
the library has been configured to use LiMIC2. 


\subsection{MV2\_USE\_MCAST}
\label{def:use-mcast}
\begin{itemize}
        \item Class: Run time
        \item Default: 0
        \item Applicable interface(s): OFA-IB-CH3
\end{itemize}
Set this to 1, to enable hardware multicast support in collective communication

\subsection{MV2\_MCAST\_NUM\_NODES\_THRESHOLD}
\label{def:mcast-num-nodes-thrshold}
\begin{itemize}
        \item Class: Run time
        \item Default: 8
        \item Applicable interface(s): OFA-IB-CH3
\end{itemize}
This defines the threshold for enabling multicast support in collective communication. 
When MV2\_USE\_MCAST is set to 1 and the number of nodes in the job is greater than or 
equal to the threshold value, it uses multicast support in collective communication

\subsection{MV2\_USE\_CUDA}
\label{def:use-cuda}

\begin{itemize}
        \item Class: Run time
        \item Default: 0
        \item Applicable interface(s): OFA-IB-CH3
\end{itemize}
set this to One. to enable support for communication with GPU device buffers.

\subsection{MV2\_CUDA\_BLOCK\_SIZE}
\label{def:cuda-block-size}

\begin{itemize}
        \item Class: Run time
        \item Default: 262144
        \item Applicable interface(s): OFA-IB-CH3
\end{itemize}
The chunk size used in large message transfer from device memory to host memory.
The other suggested values for this parameter are 131072 and 524288.

%\subsection{MV2\_CUDA\_EVENT\_SYNC}
%\label{def:cuda-event-sync}
%
%\begin{itemize}
%        \item Class: Run time
%        \item Default: 1
%        \item Applicable interface(s): OFA-IB-CH3
%\end{itemize}
%MVAPICH2 uses cuda events to synchronization in pipelined communication from device.
%Disabling this parameter enables use of cuda streams for synchronization instead of cuda events.

\subsection{MV2\_CUDA\_KERNEL\_VECTOR\_TIDBLK\_SIZE}
\label{def:cuda-kernel-vector-threadblock}

\begin{itemize}
        \item Class: Run time
        \item Default: 1024
        \item Applicable interface(s): OFA-IB-CH3
\end{itemize}
This controls the number of CUDA threads per block in pack/unpack kernels for MPI vector datatype in communication 
involving GPU device buffers.

\subsection{MV2\_CUDA\_KERNEL\_VECTOR\_YSIZE}
\label{def:cuda-kernel-vector-ysize}

\begin{itemize}
        \item Class: Run time
        \item Default: tuned based on dimensions of the vector
        \item Applicable interface(s): OFA-IB-CH3
\end{itemize}
This controls the y-dimension of a thread block in pack/unpack kernels for MPI vector datatype in communication
involving GPU device buffers. It controls the number of threads operating on each block of data in a vector.

\subsection{MV2\_CUDA\_NONBLOCKING\_STREAMS}
\label{def:cuda-nonblocking-streams}

\begin{itemize}
        \item Class: Run time
        \item Default: 1 (Enabled)
        \item Applicable interface(s): OFA-IB-CH3
\end{itemize}
This controls the use of non-blocking streams for asynchronous CUDA memory copies in communication involving GPU memory.

\subsection{MV2\_CUDA\_IPC}
\label{def:cuda-ipc}

\begin{itemize}
        \item Class: Run time
        \item Default: 1
        \item Applicable interface(s): OFA-IB-CH3
\end{itemize}
This enables intra-node GPU-GPU communication using IPC feature available from CUDA 4.1

%\subsection{MV2\_CUDA\_IPC\_THRESHOLD}
%\label{def:cuda-ipc-threshold}
%
%\begin{itemize}
%        \item Class: Run time
%        \item Default: 0 bytes
%        \item Applicable interface(s): OFA-IB-CH3
%\end{itemize}
%This specifies the threshold for the message size beyond which uses CUDA IPC feature for
%GPU-GPU communication within a node. Default, it uses IPC for all size of the message sizes.

\subsection{MV2\_CUDA\_SMP\_IPC}
\label{def:cuda-ipc}

\begin{itemize}
        \item Class: Run time
        \item Default: 0
        \item Applicable interface(s): OFA-IB-CH3
\end{itemize}
This enables an optimization for short message GPU device-to-device communication using IPC feature available from CUDA 4.1

\subsection{MV2\_ENABLE\_SHARP}
\label{def:enable-sharp}
\begin{itemize}
        \item Class: Run time
        \item Default: 0
        \item Applicable interface(s): OFA-IB-CH3
\end{itemize}
Set this to 1, to enable hardware SHArP support in collective communication

\subsection{MV2\_SHARP\_HCA\_NAME}
\label{def:sharp-hca-name}
\begin{itemize}
        \item Class: Run time
        \item Default: unset
        \item Applicable interface(s): OFA-IB-CH3
\end{itemize}
By default, this is set by the MVAPICH2 library. However, you can explicitly
set the HCA name which is realized by the SHArP library. 

\subsection{MV2\_SHARP\_PORT}
\label{def:sharp-port}
\begin{itemize}
        \item Class: Run time
        \item Default: 1
        \item Applicable interface(s): OFA-IB-CH3
\end{itemize}
By default, this is set by the MVAPICH2 library. However, you can explicitly
set the HCA port which is realized by the SHArP library. 

\subsection{MV2\_ENABLE\_SOCKET\_AWARE\_COLLECTIVES}
\label{def:mv2-enable-socket-aware-collectives}
\begin{itemize}
\item Class : Run time
\item Default : Enabled (1)
\item Applicable interface(s): OFA-IB-CH3, OFA-iWARP-CH3
\end{itemize}
This parameter enables/disables support for socket-aware collective communication. 
The parameter MV2\_USE\_SHMEM\_COLL must be set to 1 for this to work.

\subsection{MV2\_USE\_TOPO\_AWARE\_ALLREDUCE}
\label{def:mv2-use-topo-aware-allreduce}
\begin{itemize}
\item Class : Run time
\item Default : Enabled (1)
\item Applicable interface(s): OFA-IB-CH3, OFA-iWARP-CH3
\end{itemize}
This parameter determines whether a topology-aware algorithm should be used or not for 
allreduce collective operations. It takes effect only if 
MV2\_ENABLE\_TOPO\_AWARE\_COLLECTIVES is set to 1.

\subsection{MV2\_USE\_TOPO\_AWARE\_BARRIER}
\label{def:mv2-use-topo-aware-barrier}
\begin{itemize}
\item Class : Run time
\item Default : Enabled (1)
\item Applicable interface(s): OFA-IB-CH3, OFA-iWARP-CH3
\end{itemize}
This parameter determines whether a topology-aware algorithm should be used or not for 
barrier collective operations. It takes effect only if 
MV2\_ENABLE\_TOPO\_AWARE\_COLLECTIVES is set to 1.

\subsection{MV2\_USE\_RDMA\_CM\_MCAST}
\label{def:mv2-use-rdma-cm-mcast}
\begin{itemize}
\item Class : Run time
\item Default : Enabled (1)
\item Applicable interface(s): OFA-IB-CH3
\end{itemize}
This parameter enables support for RDMA\_CM based multicast group setup. Requires 
the parameter MV2\_USE\_MCAST to be set to 1.
