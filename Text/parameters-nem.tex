\section{MVAPICH2 Parameters (OFA-IB-Nemesis Interface)}
\label{def:mvapich-parameters-nem}

%\subsection{MV2\_3DTORUS\_SUPPORT}
%\label{def:mv2-nem-3dtorus-support}
%\begin{itemize}
%    \item Class: Run time
%    \item Default: Disabled
%    \item Value Domain: $<$0, 1$>$
%\end{itemize}
%When this environment variable is set, the MPI library will query the
%Subnet Manager for the correct Service Level to be used for a given path
%between a pair LIDs in a 3D torus network.

%\subsection{MV2\_ALLTOALL\_DREG\_DISABLE\_THRESHOLD}\index{MV2\_ALLTOALL\_DREG\_DISABLE\_THRESHOLD}
%\label{def:nem-mv2_alltoall_dreg_disable_threshold}
%\begin{itemize}
%    \item Class: Run Time
%    \item Default: 1024
%%    \item Applicable interface(s): Gen2, Gen2-iWARP
%\end{itemize}
%This variable denotes the largest system size
%up to which the MPI\_Alltoall operation can use the registration cache. This can be
%used for debugging purposes. Beyond this system size, the registration cache
%is selectively disabled for the Alltoall operation alone. The default value is
%set to 1024 processes. 


%\subsection{MV2\_ALLTOALL\_DREG\_DISABLE}\index{MV2\_ALLTOALL\_DREG\_DISABLE}
%\label{def:nem-mv2_alltoall_dreg_disable}
%\begin{itemize}
%    \item Class: Run Time
%    \item Default: 0
%%    \item Applicable interface(s): Gen2, Gen2-iWARP
%\end{itemize}
%Users can selectively disable the registration cache for the MPI\_Alltoall operation
%alone by setting this value to 1. 



%\subsection{MV2\_ALLTOALL\_DREG\_DISABLE\_THRESHOLD}
%\label{def:nem-mv2_alltoall_dreg_disable_threshold}
%\begin{itemize}
 %   \item Class: Run Time
%    \item Default: 1024
%    \item Applicable interface(s): Gen2, Gen2-iWARP
%\end{itemize}
%This variable denotes the largest system size
%upto which the MPI\_Alltoall operation can use the registration cache. This can be
%used for debugging purposes. Beyond these system sizes, the registration cache
%is selectively disabled for the Alltoall operation alone. The default value is
%set to 1024 processes. 


%\subsection{MV2\_ALLTOALL\_DREG\_DISABLE}
%\label{def:nem-mv2_alltoall_dreg_disable}
%\begin{itemize}
%    \item Class: Run Time
%    \item Default: 0
%%    \item Applicable interface(s): Gen2, Gen2-iWARP
%\end{itemize}
%Users can selectively disable the registration cache for the MPI\_Alltoall operation
%alone by setting this value to 1. 
%

% \subsection{Checkpoint}
% 
% \subsection{MV2\_CKPT\_FILE}
% \label{def:nem-mv2-ckpt-file}
% \begin{itemize}
%     \item Class: Run Time
%     \item Default: /tmp/ckpt
%     \item Applicable interface(s): Gen2
% \end{itemize}
% This parameter specifies the path and the base filename for
% checkpoint files of MPI processes. The checkpoint files will be
% named as \${MV2\_CKPT\_FILE.$<$number of checkpoint$>$.$<$process rank$>$},
% for example, /tmp/ckpt.1.0 is the checkpoint file for process 0's
% first checkpoint. To checkpoint on network-based file systems, user
% just need to specify the path to it, such as
% /mnt/pvfs2/my\_ckpt\_file.
% 
% \subsection{MV2\_CKPT\_INTERVAL}
% \label{def:nem-mv2-ckpt-interval}
% \begin{itemize}
%     \item Class: Run Time
%     \item Default: 0
%     \item Unit: minutes
%     \item Applicable interface(s): Gen2
% \end{itemize}
% This parameter can be used to enable automatic checkpointing. To let
% MPI job console automatically take checkpoints, this value needs to
% be set to the desired checkpointing interval. A zero will disable
% automatic checkpointing. Using automatic checkpointing, the
% checkpoint file for the MPI job console will be named as
% \${MV2\_CKPT\_FILE.$<$number of checkpoint$>$.auto}. Users need to use
% this file for restart.
% 
% \subsection{MV2\_CKPT\_MAX\_SAVE\_CKPTS}
% \label{def:nem-mv2-max-save-ckpts}
% \begin{itemize}
%     \item Class: Run Time
%     \item Default: 0
%     \item Applicable interface(s): Gen2
% \end{itemize}
% This parameter is used to limit the number of checkpoints saved on
% file system to save the file system space. When set to a positive
% value N, only the last N checkpoints will be saved.
% 
% \subsection{MV2\_CKPT\_NO\_SYNC}
% \label{def:nem-mv2-ckpt-no-sync}
% \begin{itemize}
%     \item Class: Run Time
%     \item Applicable interface(s): Gen2
% \end{itemize}
% When this parameter is set to any value, the checkpoints will not be
% required to sync to disk. It can reduce the checkpointing delay in
% many cases. But if users are using local file system, or any
% parallel file system with local cache, to store the checkpoints, it
% is recommended not to set this parameter because otherwise the
% checkpoint files will be cached in local memory and will likely be
% lost upon failure.

% \subsection{Connection manager}
% 
% \subsection{MV2\_CM\_RECV\_BUFFERS}
% \label{def:nem-mv2-cm-recv-buffers}
% \begin{itemize}
%     \item Class: Run Time
%     \item Default: 1024
% %    \item Applicable interface(s): Gen2
% \end{itemize}
% This defines the number of buffers used by connection manager to
% establish new connections. These buffers are quite small and are
% shared for all connections, so this value may be increased to 8192
% for large clusters to avoid retries in case of packet drops.
% 
% \subsection{MV2\_CM\_SPIN\_COUNT}
% \label{def:nem-mv2-cm-spin-count}
% \begin{itemize}
%     \item Class: Run Time
%     \item Default: 5000
% %    \item Applicable interface(s): Gen2
% \end{itemize}
% This is the number of the connection manager polls for new control
% messages from UD channel for each interrupt. This may be increased
% to reduce the interrupt overhead when many incoming control messages
% from UD channel at the same time.
% 
% 
% \subsection{MV2\_CM\_TIMEOUT}
% \label{def:nem-mv2-cm-timeout}
% \begin{itemize}
%     \item Class: Run Time
%     \item Default: 500
%     \item Unit: milliseconds
% %    \item Applicable interface(s): Gen2
% \end{itemize}
% This is the timeout value associated with connection management
% messages via UD channel. Decreasing this value may lead to faster
% retries but at the cost of generating duplicate messages.


% \subsection{MV2\_CPU\_MAPPING}
% \label{def:nem-mv2-cpu-mapping}
% \begin{itemize}
%     \item Class: Run Time
%     \item Default: Local rank based mapping
% %    \item Applicable interface(s): Gen2, Gen2-iWARP, uDAPL (Linux)
% \end{itemize}
% 
% This allows users to specify process to CPU (core) mapping. The detailed usage
% of this parameter is described in Section~\ref{usage:mv2-cpu-mapping}. This 
% parameter will not take effect if \texttt{MV2\_ENABLE\_AFFINITY} is set to 0.
% MV2\_CPU\_MAPPING is currently not supported on Solaris.
% 
% 

\subsection{MV2\_DEFAULT\_MAX\_SEND\_WQE}
\label{def:nem-max-send-wqe}

\begin{itemize}
        \item Class: Run time
        \item Default: 64
%	\item Applicable interface(s): Gen2, Gen2-iWARP, uDAPL
\end{itemize}

This specifies the maximum number of send WQEs on each QP. 
Please note that for Gen2 and Gen2-iWARP, the default value of this parameter
will be 16 if the number of processes is larger than 256 for better
memory scalability.

\subsection{MV2\_DEFAULT\_MAX\_RECV\_WQE}
\label{def:nem-max-recv-wqe}

\begin{itemize}
        \item Class: Run time
        \item Default: 128
%    \item Applicable interface(s): Gen2, Gen2-iWARP, uDAPL
\end{itemize}

This specifies the maximum number of receive WQEs on each QP (maximum
number of receives that can be posted on a single QP). 

\subsection{MV2\_DEFAULT\_MTU}
\label{def:nem-rdma-default-mtu}

\begin{itemize}
    \item Class: Run time

    \item Default: IBV\_MTU\_1024 for IB SDR cards and IBV\_MTU\_2048 for IB DDR and
		QDR cards.
%    \item Applicable interface(s): Gen2, uDAPL
\end{itemize}

The internal MTU size. For Gen2, this parameter should be a string instead
of an integer. Valid values are: \texttt{IBV\_MTU\_256}, \texttt{IBV\_MTU\_512},
\texttt{IBV\_MTU\_1024}, \texttt{IBV\_MTU\_2048}, \texttt{IBV\_MTU\_4096}.

\subsection{MV2\_DEFAULT\_PKEY}
\label{def:nem-mv2-default-pkey}
\begin{itemize}
		\item Class: Run Time
%		\item Applicable Interface(s): Gen2
\end{itemize}

Select the partition to be used for the job.

%\subsection{MV2\_ENABLE\_AFFINITY}
%\label{def:nem-viadev_enable_affinity}
%
%\begin{itemize}
%    \item Class: Run time
%    \item Default: 1
% %   \item Applicable interface(s): Gen2, Gen2-iWARP, uDAPL (Linux)
%\end{itemize}

%Enable CPU affinity by setting MV2\_ENABLE\_AFFINITY to 1 or disable it by
%setting \\
%MV2\_ENABLE\_AFFINITY to 0. MV2\_ENABLE\_AFFINITY is currently not supported
%on Solaris.


%\subsection{MV2\_GET\_FALLBACK\_THRESHOLD}
%\begin{itemize}
%        \item Class: Run time
%        \item This threshold value needs to be set in bytes.
%        \item This option is effective if we define ONE\_SIDED flag.
%%    \item Applicable interface(s): Gen2, Gen2-iWARP, uDAPL
%
%\end{itemize}
%This defines the threshold beyond which the MPI\_Get implementation is based on direct one sided RDMA operations.


\subsection{MV2\_IBA\_EAGER\_THRESHOLD}
\label{def:nem-rdma-iba-eager-threshold}

\begin{itemize}
    \item Class: Run time

    \item Default: Architecture dependent (12KB for IA-32)
%    \item Applicable interface(s): Gen2, Gen2-iWARP, uDAPL
\end{itemize}

This specifies the switch point between eager and rendezvous
protocol in MVAPICH2. For better performance, the value of 
MV2\_IBA\_EAGER\_THRESHOLD should be
set the same as MV2\_VBUF\_TOTAL\_SIZE.

\subsection{MV2\_IBA\_HCA}
\label{def:nem-rdma-iba-hca}

\begin{itemize}
    \item Class: Run time

    \item Default: Unset
%    \item Applicable interface(s): Gen2, Gen2-iWARP, uDAPL
\end{itemize}

This specifies the HCA to be used for performing network operations.

\subsection{MV2\_INITIAL\_PREPOST\_DEPTH}
\label{def:nem-viadev-initial-prepost-depth}

\begin{itemize}
    \item Class: Run time

    \item Default: 10
%    \item Applicable interface(s): Gen2, Gen2-iWARP, uDAPL
\end{itemize}

This defines the initial number of pre-posted receive buffers for each
connection. If communication happen for a particular connection, the
number of buffers will be increased to \\RDMA\_PREPOST\_DEPTH.

% \subsection{Knomial}
% 
% \subsection{MV2\_KNOMIAL\_INTRA\_NODE\_FACTOR}
% \label{def:nem-mv2_knomial_intra_node_factor}
% \begin{itemize}
%     \item Class: Run time
%     \item Default: 4
% %    \item Applicable interface(s): Gen2, Gen2-iWARP, uDAPL
% \end{itemize}
% 
% This defines the degree of the knomial operation during the intra-node
% knomial broadcast phase. 
% 
% \subsection{MV2\_KNOMIAL\_INTER\_NODE\_FACTOR}
% \label{def:nem-mv2_knomial_inter_node_factor}
% \begin{itemize}
%     \item Class: Run time
%     \item Default: 4
% %    \item Applicable interface(s): Gen2, Gen2-iWARP, uDAPL
% \end{itemize}
% 
% This defines the degree of the knomial operation during the inter-node
% knomial broadcast phase. 
% 
% 


\subsection{MV2\_MAX\_INLINE\_SIZE}
\label{def:nem-max-inline-size}
\begin{itemize}
    \item Class: Run time
    \item Default: Network card dependent (128 for most networks including InfiniBand)
%    \item Applicable interface(s): Gen2, Gen2-iWARP
\end{itemize}

This defines the maximum inline size for data transfer. Please note that the 
default value of this parameter will be 0 when the number of processes is larger than
256 to improve memory usage scalability.

\subsection{MV2\_NDREG\_ENTRIES}
\label{def:nem-ndreg-entries}
\begin{itemize}
    \item Class: Run time
    \item Default: 1000
%    \item Applicable interface(s): Gen2, Gen2-iWARP, uDAPL
\end{itemize}

This defines the total number of buffers that can be stored in the
registration cache. It has no effect if MV2\_USE\_LAZY\_MEM\_UNREGISTER is
not set. A larger value will lead to less frequent lazy
de-registration.


%\subsection{MV2\_NUM\_HCAS}
%\label{def:nem-num-hcas}
%\begin{itemize}
%    \item Class: Run time
%    \item Default: 1
%%    \item Applicable interface(s): Gen2, Gen2-iWARP
%\end{itemize}
%This parameter indicates number of InfiniBand adapters to be used for communication
%on an end node.
%
%\subsection{MV2\_NUM\_PORTS}
%\label{def:nem-num-ports}
%\begin{itemize}
%    \item Class: Run time
%    \item Default: 1
%%    \item Applicable interface(s): Gen2, Gen2-iWARP
%\end{itemize}
%This parameter indicates number of ports per InfiniBand adapter to be used for communication per adapter on an end node.
%
%
%\subsection{MV2\_NUM\_QP\_PER\_PORT}
%\label{def:nem-num-qp-per-port}
%\begin{itemize}
%    \item Class: Run time
%    \item Default: 1
%%    \item Applicable interface(s): Gen2, Gen2-iWARP
%\end{itemize}
%This parameter indicates number of queue pairs
%per port to be used for communication on an end node.
%This is useful in the presence of multiple send/recv engines
%available per port for data transfer.


\subsection{MV2\_NUM\_RDMA\_BUFFER}
\label{def:nem-num-rdma-buffer}

\begin{itemize}
    \item Class: Run time

    \item Default: Architecture dependent (32 for EM64T)
%    \item Applicable interface(s): Gen2, Gen2-iWARP, uDAPL
\end{itemize}

The number of RDMA buffers used for the RDMA fast path. This \emph{fast
path} is used to reduce latency and overhead of small data and control
messages. This value will be ineffective if MV2\_USE\_RDMA\_FAST\_PATH is
not set. 

\subsection{MV2\_NUM\_SA\_QUERY\_RETRIES}
\label{def:mv2_num_sa_query_retries}
\begin{itemize}
    \item Class: Run time
    \item Default: 20
    \item Applicable Interface(s): OFA-IB-CH3, OFA-iWARP-CH3
\end{itemize}
Number of times the MPI library will attempt to query the subnet to obtain 
the path record information before giving up.

% \subsection{MV2\_ON\_DEMAND\_THRESHOLD}
% \label{def:nem-mv2-on-demand-threshold}
% \begin{itemize}
%     \item Class: Run Time
%     \item Default: 64 (Gen2, uDAPL), 16 (Gen2-iWARP)
% %    \item Applicable interface(s): Gen2, Gen2-iWARP, uDAPL
% \end{itemize}
% This defines threshold for enabling on-demand connection management
% scheme. When the size of the job is larger than the threshold value,
% on-demand connection management will be used.

%\subsection{MV2\_PATH\_SL\_QUERY}
%\label{def:nem-path-sl-query}
%\begin{itemize}
%    \item Class: Run time
%    \item Default: Disabled
%    \item Value Domain: $<$0, 1$>$
%    \item Applicable interface(s): OFA-IB-CH3, OFA-iWARP-CH3
%\end{itemize}
%When this environment variable is set, the MPI library will query the
%Subnet Manager for the correct Service Level to be used for a given path
%between a pair LIDs in the network.

\subsection{MV2\_PREPOST\_DEPTH}
\label{def:nem-rdma-prepost-depth}

\begin{itemize}
    \item Class: Run time

    \item Default: 64
%    \item Applicable interface(s): Gen2, Gen2-iWARP, uDAPL
\end{itemize}

This defines the number of buffers pre-posted for each connection to
handle send/receive operations.

% \subsection{MV2\_PSM\_DEBUG}
% \label{def:nem-psm-debug}
% 
% \begin{itemize}
%         \item Class: Run time (Debug)
%         \item Default: 0
% %    \item Applicable interface: PSM
% \end{itemize}
% 
% This parameter enables the dumping of run-time debug counters from the
% MVAPICH2-PSM progress engine. Counters are dumped every PSM\_DUMP\_FREQUENCY
% seconds.
% 
% \subsection{MV2\_PSM\_DUMP\_FREQUENCY}
% \label{def:nem-psm-dump}
% 
% \begin{itemize}
%         \item Class: Run time (Debug)
%         \item Default: 10 seconds
% %    \item Applicable interface: PSM
% \end{itemize}
% 
% This parameters sets the frequency for dumping MVAPICH2-PSM debug counters.
% Value takes effect only in PSM\_DEBUG is enabled.

%\subsection{MV2\_PUT\_FALLBACK\_THRESHOLD}
%\begin{itemize}
%        \item Class: Run time
%        \item This threshold value needs to be set in bytes.
%        \item This option is effective if we define ONE\_SIDED flag.
%%    \item Applicable interface(s): Gen2, Gen2-iWARP, uDAPL
%
%\end{itemize}
%This defines the threshold beyond which the MPI\_Put implementation is based on
%direct one sided RDMA operations.

% \subsection{MV2\_RDMA\_CM\_ARP\_TIMEOUT}
% \label{def:nem-mv2-rdma-cm-arp-timeout}
% \begin{itemize}
%     \item Class: Run Time
%     \item Default: 2000 ms
% %    \item Applicable interface(s): Gen2, Gen2-iWARP
% \end{itemize}
% This parameter specifies the arp timeout to be used by RDMA CM module.
% 
% \subsection{MV2\_RDMA\_CM\_MAX\_PORT}
% \label{def:nem-mv2-rdma-cm-max-port}
% \begin{itemize}
%     \item Class: Run Time
%     \item Default: Unset
% %    \item Applicable interface(s): Gen2, Gen2-iWARP
% \end{itemize}
% This parameter specifies the upper limit of the port range to be used 
% by the RDMA CM module when choosing the port on which it listens for 
% connections.
% 
% \subsection{MV2\_RDMA\_CM\_MIN\_PORT}
% \label{def:nem-mv2-rdma-cm-min-port}
% \begin{itemize}
%     \item Class: Run Time
%     \item Default: Unset
% %    \item Applicable interface(s): Gen2, Gen2-iWARP
% \end{itemize}
% This parameter specifies the lower limit of the port range to be used 
% by the RDMA CM module when choosing the port on which it listens for 
% connections.

\subsection{MV2\_RNDV\_PROTOCOL}
\begin{itemize}
    \item Class: Run time
    \item Default: RPUT
%    \item Applicable interface(s): Gen2, Gen2-iWARP
\end{itemize}
The value of this variable can be set to choose different Rendezvous
protocols. RPUT (default RDMA-Write) RGET (RDMA Read based), R3
(send/recv based).

\subsection{MV2\_R3\_THRESHOLD}
\begin{itemize}
    \item Class: Run time
    \item Default: 4096
%    \item Applicable interface(s): Gen2, Gen2-iWARP
\end{itemize}

The value of this variable controls what message sizes go over the 
R3 rendezvous protocol. Messages above this message size use
MV2\_RNDV\_PROTOCOL. 

\subsection{MV2\_R3\_NOCACHE\_THRESHOLD}
\begin{itemize}
    \item Class: Run time
    \item Default: 32768
%    \item Applicable interface(s): Gen2, Gen2-iWARP
\end{itemize}

The value of this variable controls what message sizes go over the 
R3 rendezvous protocol when the registration cache is disabled (MV2\_USE\_LAZY\_MEM\_UNREGISTER=0). Messages above this message size use
MV2\_RNDV\_PROTOCOL. 

% \subsection{MV2\_SHMEM\_ALLREDUCE\_MSG}
% \label{def:nem-mv2-shmem-coll-allreduce-threshold}
% \begin{itemize}
%     \item Class: Run Time
%     \item Default: 1 $\ll$ 15
%     \item Applicable interface(s): Gen2, Gen2-iWARP
% \end{itemize}
% 
% The SHMEM allreduce is used for messages less than this threshold.
% 
% \subsection{MV2\_SHMEM\_BCAST\_LEADERS}
% \label{def:nem-mv2-shmem-bcast-leaders}
% 
% \begin{itemize}
%         \item Class: Run time
%         \item Default: 4096
% \end{itemize}
% 
% The number of leader processes that will take part in the SHMEM
% broadcast operation. Must be greater than the number of nodes in the
% job.
% 
% \subsection{MV2\_SHMEM\_BCAST\_MSG}
% \label{def:nem-mv2-shmem-coll-bcast-threshold}
% \begin{itemize}
%     \item Class: Run Time
%     \item Default: 1 $\ll$ 20
%     \item Applicable interface(s): Gen2, Gen2-iWARP
% \end{itemize}
% The SHMEM bcast is used for messages less than this threshold.
% 
% \subsection{MV2\_SHMEM\_COLL\_MAX\_MSG\_SIZE}
% \label{def:nem-shmem-coll-max-msg-size}
% \begin{itemize}
%     \item Class: Run Time
%     \item Applicable interface(s): Gen2, Gen2-iWARP
% \end{itemize}
% This parameter can be used to select the max buffer size of message
% for shared memory collectives.
% 
% \subsection{MV2\_SHMEM\_COLL\_NUM\_COMM}
% \label{def:nem-shmem-coll-num-comm}
% \begin{itemize}
%     \item Class: Run Time
%     \item Applicable interface(s): Gen2, Gen2-iWARP
% \end{itemize}
% This parameter can be used to select the number of communicators
% using shared memory collectives.
% 
% \subsection{MV2\_SHMEM\_DIR}
% \label{def:nem-shmem-dir}
% \begin{itemize}
%     \item Class: Run Time
%     \item Applicable interface(s): Gen2, Gen2-iWARP, uDAPL
%     \item Default: /dev/shm for Linux and /tmp for Solaris
% \end{itemize}
% This parameter can be used to specify the path to the shared memory
% files for intra-node communication.
% 
% \subsection{MV2\_SHMEM\_REDUCE\_MSG}
% \label{def:nem-mv2-shmem-coll-reduce-threshold}
% \begin{itemize}
%     \item Class: Run Time
%     \item Default: 1 $\ll$ 10
%     \item Applicable interface(s): Gen2, Gen2-iWARP
% \end{itemize}
% The SHMEM reduce is used for messages less than this threshold.
% 

% \subsection{MV2\_SM\_SCHEDULING}
% \label{def:nem-mv2-sm-scheduling}
% \begin{itemize}
%     \item Class: Run Time
%     \item Default: USE\_FIRST (Options: ROUND\_ROBIN)
% %    \item Applicable interface(s): Gen2, Gen2-iWARP
% \end{itemize}
% 
% \subsection{MV2\_SMP\_USE\_LIMIC2}
% \label{def:nem-mv2-smp-limic2}
% \begin{itemize}
%     \item Class: Run Time
%     \item Default: On if configured with --with-limic2
% %    \item Applicable interface(s): Gen2, Gen2-iWARP, uDAPL
% \end{itemize}
% 
% This parameter enables/disables LiMIC2 at run time. It does
% not take effect if MVAPICH2 is not configured with --with-limic2.

\subsection{MV2\_SRQ\_LIMIT}
\label{def:nem-viadev-srq-limit}

\begin{itemize}
    \item Class: Run Time
    \item Default: 30
%    \item Applicable interface(s): Gen2, Gen2-iWARP
\end{itemize}

This is the low water-mark limit for the
Shared Receive Queue. If the
number of available work entries on the
SRQ drops below this limit, the
flow control will be activated.


\subsection{MV2\_SRQ\_SIZE}
\label{def:nem-viadev-srq-size}

\begin{itemize}
    \item Class: Run Time
    \item Default: 512
%    \item Applicable interface(s): Gen2, Gen2-iWARP
\end{itemize}

This is the maximum number of work
requests allowed on the Shared
Receive Queue.

\subsection{MV2\_STRIPING\_THRESHOLD}
\label{def:nem-viadev-striping-threshold}

\begin{itemize}
    \item Class: Run Time
    \item Default: 8192
%    \item Applicable interface(s): Gen2, Gen2-iWARP
\end{itemize}

This parameter specifies the message size above which we begin the stripe the
message across multiple rails (if present).

% \subsection{MV2\_SUPPORT\_DPM}
% \label{def:nem-support-dpm}
% 
% \begin{itemize}
% 	\item Class: Run time
% 	\item Default: 0 (disabled)
% 	\item Applicable interface: Gen2
% \end{itemize}
% 
% This option enables the dynamic process management interface and on-demand connection
% management.

%\subsection{MV2\_USE\_APM}
%\label{def:nem-mv2-use-apm}
%%\begin{itemize}
%%	\item Class: Run Time
%%	\item Applicable interface(s): Gen2
%%\end{itemize}
%
%This parameter is used for recovery from network faults using Automatic
%Path Migration. This functionality is beneficial in the presence of
%multiple paths in the network, which can be enabled by using lmc
%mechanism. 

%\subsection{MV2\_USE\_APM\_TEST}
%\label{def:nem-mv2-use-apm-test}
%% \begin{itemize}
%% 	\item Class: Run Time
%% 	\item Applicable interface(s): Gen2
%% \end{itemize}
%
%This parameter is used for testing the Automatic Path Migration
%functionality. It periodically moves the alternate path as the primary
%path of communication and re-loads another alternate path.

\subsection{MV2\_USE\_BLOCKING}
\begin{itemize}
    \item Class: Run time
    \item Default: 0
%     \item Applicable interface(s): Gen2
\end{itemize}
Setting this parameter enables mvapich2 to use blocking mode progress.
MPI applications do not take up any CPU when they are waiting for
incoming messages.


% \subsection{MV2\_USE\_COALESCE}
% \begin{itemize}
%     \item Class: Run time
%     \item Default: set
% %    \item Applicable interface(s): Gen2, Gen2-iWARP
% \end{itemize}
% Setting this parameter enables message coalescing to increase small
% message throughput 
% 

%\subsection{MV2\_USE\_HSAM}
%\label{def:nem-mv2-use-hsam}
%% \begin{itemize}
%% 	\item Class: Run Time
%% 	\item Applicable interface(s): Gen2
%% \end{itemize}
%
%This parameter is used for utilizing hot-spot avoidance with InfiniBand
%clusters. To leverage this functionality, the subnet should be
%configured with lmc greater than zero. Please refer to
%section~\ref{def:mv2-hsam} for detailed information.

% \subsection{MV2\_USE\_HWLOC\_CPU\_BINDING}
% \label{def:nem-mv2-use-hwloc-cpu-binding}
% \begin{itemize}
%     \item Class: Run Time
%     \item Default: On
% %    \item Applicable interface(s): Gen2, Gen2-iWARP, uDAPL (Linux)
% \end{itemize}
% 
% The library needs to be configured with the --with-hwloc flag to get the
% benefits of the HWLOC package. Users can choose to disable
% hwloc feature at run-time, by setting MV2\_USE\_HWLOC\\
% \_CPU\_BINDING = 0.
% The various options related to choosing the CPU mapping patterns were discussed
% in Section \ref{usage:mv2-cpu-mapping}.
% 
% 
% \subsection{MV2\_USE\_IWARP\_MODE}
% \label{def:nem-mv2-enable-iwarp-mode}
% \begin{itemize}
%     \item Class: Run Time
%     \item Default: unset
% %    \item Applicable interface(s): Gen2, Gen2-iWARP
% \end{itemize}
% This parameter enables the library to run in iWARP mode. The library
% has to be built using the flag -DRDMA\_CM for using this feature.
% 

\subsection{MV2\_USE\_LAZY\_MEM\_UNREGISTER}
\begin{itemize}
    \item Class: Run time
    \item Default: set
%    \item Applicable interface(s): Gen2, Gen2-iWARP, uDAPL
\end{itemize}
Setting this parameter enables mvapich2 to use memory registration cache.


%\subsection{MV2\_USE\_RoCE}
%\label{def:nem-mv2-use-roce}
%\begin{itemize}
%    \item Class: Run Time
%    \item Default: Un Set
%%    \item Applicable interface(s): Gen2
%\end{itemize}
%This parameter enables the use of RDMA over Ethernet for MPI communication.
%The underlying HCA and network must support this feature.

% \subsection{MV2\_USE\_RDMA\_CM}
% \label{def:nem-mv2-use-rdma-cm}
% \begin{itemize}
%     \item Class: Run Time
%     \item Default: Network Dependant (set for Gen2-iWARP and unset for Gen2)
% %    \item Applicable interface(s): Gen2, Gen2-iWARP
% \end{itemize}
% This parameter enables the use of RDMA CM for establishing the
% connections. The library has to be built using the flag -DRDMA\_CM for
% using this feature.
% 
\subsection{MV2\_USE\_RDMA\_FAST\_PATH}
\label{def:nem-disable-rfp}
\begin{itemize}
    \item Class: Run time
    \item Default: set
%    \item Applicable interface(s): Gen2, Gen2-iWARP, uDAPL
\end{itemize}
Setting this parameter enables MVAPICH2 to use adaptive rdma fast path features
for the Gen2 interface.



%\subsection{MV2\_USE\_RDMA\_ONE\_SIDED}
%\begin{itemize}
%    \item Class: Run time
%    \item Default: set
%%    \item Applicable interface(s): Gen2, Gen2-iWARP, uDAPL
%\end{itemize}
%Setting this parameter allows mvapich2 to use optimized one sided implementation
%based RDMA operations.


% \subsection{MV2\_USE\_RING\_STARTUP}
% \begin{itemize}
%     \item Class: Run time
%     \item Default: set
% %    \item Applicable interface(s): Gen2
% \end{itemize}
% Setting this parameter enables mvapich2 to use ring based startup. 
% 
% 

% \subsection{MV2\_USE\_SHARED\_MEM}
% \label{def:nem-use-shared-mem}
% \begin{itemize}
%     \item Class: Run time
%     \item Default: set 
%     \item Applicable interface(s): Gen2, Gen2-iWARP, uDAPL
% \end{itemize}
% 
% Use shared memory for intra-node communication.
% 
% \subsection{MV2\_USE\_SHMEM\_ALLREDUCE}
% \label{def:nem-mv2-use-shmem-allreduce}
% \begin{itemize}
% 		\item Class: Run Time
% 		\item Applicable interface(s): Gen2, Gen2-iWARP, uDAPL,
% 			VAPI
% \end{itemize}
% This parameter can be used to turn off shared memory based
% MPI\_Allreduce for Gen2 over IBA by setting this to 0.
% 
% 
% \subsection{MV2\_USE\_SHMEM\_BARRIER}
% \label{def:nem-mv2-use-shmem-barrier}
% \begin{itemize}
% 		\item Class: Run Time
% 		\item Applicable interface(s): Gen2, Gen2-iWARP, uDAPL,
% 			VAPI
% \end{itemize}
% This parameter can be used to turn off shared memory based
% MPI\_Barrier for Gen2 over IBA by setting this to 0.
% 
% \subsection{MV2\_USE\_SHMEM\_BCAST}
% \label{def:nem-mv2-use-shmem-bcast}
% \begin{itemize}
%                 \item Class: Run Time
%                 \item Applicable interface(s): Gen2, Gen2-iWARP, uDAPL
% \end{itemize}
% This parameter can be used to turn off shared memory based
% MPI\_Bcast for Gen2 over IBA by setting this to 0.
% 
% \subsection{MV2\_USE\_SHMEM\_COLL}
% \label{def:nem-mv2-use-shmem-coll}
% \begin{itemize}
%     \item Class: Run time
%     \item Default: set 
%     \item Applicable interface(s): Gen2, Gen2-iWARP, uDAPL
% \end{itemize}
% 
% Use shared memory for collective communication. Set this to 0 for
% disabling shared memory collectives.
% 
% 
% \subsection{MV2\_USE\_SHMEM\_REDUCE}
% \label{def:nem-mv2-use-shmem-reduce}
% \begin{itemize}
% 		\item Class: Run Time
% 		\item Applicable interface(s): Gen2, Gen2-iWARP, uDAPL,
% 			VAPI
% \end{itemize}
% This parameter can be used to turn off shared memory based
% MPI\_Reduce for Gen2 over IBA by setting this to 0.


\subsection{MV2\_USE\_SRQ}
\begin{itemize}
    \item Class: Run time
    \item Default: set
%    \item Applicable interface(s): Gen2, Gen2-iWARP
\end{itemize}
Setting this parameter enables mvapich2 to use shared receive queue.

% \subsection{MV2\_USE\_XRC}
% \label{def:nem-mv2_use_xrc}
% \begin{itemize}
%     \item Class: Run time
%     \item Default: 0
% %	\item Applicable Interface(s): Gen2
% \end{itemize}
% 
% Use the XRC InfiniBand transport available since Mellanox ConnectX adapters.
% This features requires OFED version later than 1.3. It also automatically
% enables SRQ and ON-DEMAND connection management. Note that the MVAPICH2
% library needs to have been configured with --enable-xrc=yes to use this 
% feature.

\subsection{MV2\_VBUF\_POOL\_SIZE}
\label{def:nem-rdma-vbuf-pool-size}

\begin{itemize}
    \item Class: Run time
    \item Default: 512
%    \item Applicable interface(s): Gen2, Gen2-iWARP
\end{itemize}

The number of vbufs in the initial pool. This pool is shared among all
the connections.


\subsection{MV2\_VBUF\_SECONDARY\_POOL\_SIZE}
\label{def:nem-rdma-vbuf-secondary-pool-size}

\begin{itemize}
    \item Class: Run time
    \item Default: 128
%    \item Applicable interface(s): Gen2, Gen2-iWARP
\end{itemize}

The number of vbufs allocated each time when the global pool is running
out in the initial pool. This is also shared among all the connections.

\subsection{MV2\_VBUF\_TOTAL\_SIZE}
\label{def:nem-vbuf-total-size}

\begin{itemize}
    \item Class: Run time

    \item Default: Architecture dependent (6 KB for EM64T)
%    \item Applicable interface(s): Gen2, Gen2-iWARP
\end{itemize}

The size of each \texttt{vbuf}, the basic communication buffer of MVAPICH2.
For better performance, the value of MV2\_IBA\_EAGER\_THRESHOLD should be
set the same as MV2\_VBUF\_TOTAL\_SIZE.

%\subsection{\_AFFINITY\_}
%\label{def:nem-_affinity_}
%
%\begin{itemize}
%    \item Class: Compile time
%    \item Default: Defined by default on Opteron machines when \_SMP\_
%    is defined. Not defined by default on other platforms or when
%    \_SMP\_ is not defined.
%    \item Applicable interface(s): Gen2, Gen2-iWARP, uDAPL
%\end{itemize}

%Use CPU affinity to improve data locality on NUMA based platforms.
%CPU affinity can be enabled/disabled by the run time parameter
%MV2\_ENABLE\_AFFINITY when \_AFFINITY\_ is defined.


%\subsection{SMP\_EAGERSIZE}
%\label{def:nem-smp-eagersize}
%
%\begin{itemize}
%    \item Class: Run time
%    \item Default: Architecture dependent
%%    \item Applicable interface(s): Gen2, Gen2-iWARP, uDAPL
%\end{itemize}
%
%This parameter defines the
%switch point from Eager protocol to Rendezvous protocol for intra-node
%communication. Note that this variable should be set in KBytes.


%\subsection{SMP\_QUEUE\_LENGTH}
%\label{def:nem-smp-queue-length}
%
%\begin{itemize}
%    \item Class: Run time
%    \item Default: Architecture dependent
%%    \item Applicable interface(s): Gen2, Gen2-iWARP, uDAPL
%\end{itemize}
%
%This parameter defines the size
%of shared buffer between every two processes on the same node for transferring
%messages smaller than or equal to SMP\_EAGERSIZE. Note that
%this variable should be set in KBytes.


%\subsection{SMP\_NUM\_SEND\_BUFFER}
%\label{def:nem-smp-num-send-buffer}
%
%\begin{itemize}
%        \item Class: Run time
%        \item Default: Architecture dependent
%%    \item Applicable interface(s): Gen2, Gen2-iWARP, uDAPL
%\end{itemize}
%
%This parameter defines the
%number of internal send buffers for sending intra-node messages larger
%than SMP\_EAGERSIZE.
%



%\subsection{SMP\_SEND\_BUF\_SIZE}
%\label{def:nem-smp-send-buf-size}
%
%\begin{itemize}
%        \item Class: Compile time
%        \item Default: Architecture dependent
%%    \item Applicable interface(s): Gen2, Gen2-iWARP, uDAPL
%\end{itemize}
%
%This parameter defines the
%packet size when sending intra-node messages larger than SMP\_EAGERSIZE.
%
%
\subsection{MV2\_RUN\_THROUGH\_STABILIZATION}
\label{def:nem-run-through-stab}

\begin{itemize}
    \item Class: Run Time
    \item Default: 0
%    \item Applicable interface(s): Gen2, Gen2-iWARP
\end{itemize}
This enables run through stabilization support to handle the process failures.
This is valid only with Hydra process manager with --disable-auto-cleanup flag.
