\section{MVAPICH2 General Parameters}
\label{def:general-parameters}

\subsection{MV2\_IGNORE\_SYSTEM\_CONFIG}
\label{def:ignore-system-config}
\begin{itemize}
    \item Class: Run time
    \item Default: 0
\end{itemize}
If set, the system configuration file is not processed.

\subsection{MV2\_IGNORE\_USER\_CONFIG}
\label{def:ignore-user-config}
\begin{itemize}
    \item Class: Run time
    \item Default: 0
\end{itemize}
If set, the user configuration file is not processed.

\subsection{MV2\_USER\_CONFIG}
\label{def:user-config}
\begin{itemize}
    \item Class: Run time
    \item Default: Unset
\end{itemize}
Specify the path of a user configuration file for mvapich2.  If this is not set the default path of ``~/.mvapich2.conf'' is used.

\subsection{MV2\_DEBUG\_CORESIZE}
\label{def:debug-coresize}
\begin{itemize}
    \item Class: Run time
    \item Default: Unset
    \item Possible values: Positive integer or "unlimited"
\end{itemize}

Set the limit for the core size resource. It allows to specify the maximum size for a core dump to be generated. It only set the soft limit and it has the respect the hard value set on the nodes.

It is similar to the \texttt{ulimit -c <coresize>} that can be run in the shell, but this will only apply to the MVAPICH2 processes (MPI processes, mpirun\_rsh, mpispawn).

Examples:
\begin{itemize}
  \item '\texttt{MV2\_DEBUG\_CORESIZE=0}' will disable core dumps for MVAPICH2 processes.
  \item '\texttt{MV2\_DEBUG\_CORESIZE=unlimited}' will enable core dumps for MVAPICH2 processes.
\end{itemize}



\subsection{MV2\_DEBUG\_SHOW\_BACKTRACE}
\label{def:debug-backtrace}
\begin{itemize}
    \item Class: Run time
    \item Default: 0 (disabled)
    \item Possible values: 1 to enable, 0 to disable
\end{itemize}

Show a backtrace when a process fails on errors like "Segmentation faults", "Bus error", "Illegal Instruction", "Abort" or "Floating point exception".

If your application uses the static version of the MVAPICH2  library, you have to link your application with the \texttt{-rdynamic} flag in order to see the function names in the backtrace. For more information, see the backtrace manpage.

\subsection{MV2\_SHOW\_ENV\_INFO}
\label{def:show-env_info}
\begin{itemize}
    \item Class: Run time
    \item Default: 0 (disabled)
    \item Possible values: 1 (short list), 2(full list)
\end{itemize}

Show the values assigned to the run time environment parameters

\subsection{MV2\_SHOW\_CPU\_BINDING}
\label{def:show-cpu-binding}
\begin{itemize}
    \item Class: Run time
    \item Default: 0 (disabled)
    \item Possible values: 1 (Show only on node containing rank 0), 2 (Show on all nodes)
\end{itemize}

If set to 1, it shows the CPU mapping of all processes on node where rank 0
exists.  If set to 2, it shows the CPU mapping of all processes on all nodes.

