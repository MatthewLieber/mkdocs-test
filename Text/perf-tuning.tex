\section{Performance Tuning}
\label{sec:performance-tuning}

MVAPICH2 supports many different parameters for tuning performance for a
wide variety of applications. These parameters can be either compile time
parameters or run time parameters.

In this section we classify these parameters depending on what you are tuning for 
and provide guidelines on how to use them.

\subsection{Two-sided Point-to-point Tuning}

Two-sided point-to-point latency and bandwidth can be tuned very simply by using the
parameters RDMA\_IBA\_EAGER\_THRESHOLD (run time parameter) and
NUM\_RDMA\_BUFFER (run time parameter).

Messages larger than RDMA\_IBA\_EAGER\_THRESHOLD will go over the
Rendezvous protocol using zero copy. While this can reduce the number of
copies, it can be costly for small messages.

NUM\_RDMA\_BUFFER indicates the number of RDMA buffers per
connection. If this parameter is increased, more outstanding messages
can be transferred by using the fast path. However, increasing this
parameter also leads to increased memory usage.

\subsection{One-sided Point-to-point Tuning}

One-sided point-to-point performance can be tuned by using the following parameters:

 \begin{itemize}
\item RDMA\_PUT\_FALLBACK\_THRESHOLD (run time parameter)
\item RDMA\_GET\_FALLBACK\_THRESHOLD (run time parameter)
\item RDMA\_EAGERSIZE\_1SC (run time parameter)
 \end{itemize}

These parameters take effect only if MVAPICH2 is
configured with -D\_1SC\_ flag.

MVAPICH2 uses two schemes for one sided communication. One is implemented
on top of point to point communication (RDMA channel) and the other is direct
one sided implementation (extension of CH3 interface). The RDMA\_PUT\_FALLBACK\_THRESHOLD
variable is
used to define the switch point between these two schemes for MPI\_PUT messages.
Messages with size smaller than this threshold will use point-to-point scheme.
Similarly, RDMA\_GET\_FALLBACK\_THRESHOLD defines the switch point
between these two schemes for MPI\_GET messages.

One sided implementation will either do copy-and-send or register user
buffer on the fly when sending messages. The variable 
RDMA\_EAGERSIZE\_1SC defines the switch point
between these two schemes. Please note that if the value 
of RDMA\_EAGERSIZE\_1SC is smaller
than\\ RDMA\_PUT\_FALLBACK\_THRESHOLD or RDMA\_GET\_FALLBACK\_THRESHOLD, it
will take no effect.


\subsection{Tuning Memory Usage}

Memory usage often plays a significant role in application performance,
and especially more so for large scale clusters. 
The main parameters which decide the memory usage are : 
 \begin{itemize}
\item VBUF\_TOTAL\_SIZE (compile time parameter)
\item NUM\_RDMA\_BUFFER (run time parameter)
 \end{itemize}

The product of VBUF\_TOTAL\_SIZE and 
NUM\_RDMA\_BUFFER generally is a measure of the amount of memory
to be pinned down for eager message passing. These buffers are not
shared across connections.
